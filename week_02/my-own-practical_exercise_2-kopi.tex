% Options for packages loaded elsewhere
\PassOptionsToPackage{unicode}{hyperref}
\PassOptionsToPackage{hyphens}{url}
%
\documentclass[
]{article}
\title{practical\_exercise\_2, Methods 3, 2021, autumn semester}
\author{Caroline Hommel}
\date{29th of September 2021}

\usepackage{amsmath,amssymb}
\usepackage{lmodern}
\usepackage{iftex}
\ifPDFTeX
  \usepackage[T1]{fontenc}
  \usepackage[utf8]{inputenc}
  \usepackage{textcomp} % provide euro and other symbols
\else % if luatex or xetex
  \usepackage{unicode-math}
  \defaultfontfeatures{Scale=MatchLowercase}
  \defaultfontfeatures[\rmfamily]{Ligatures=TeX,Scale=1}
\fi
% Use upquote if available, for straight quotes in verbatim environments
\IfFileExists{upquote.sty}{\usepackage{upquote}}{}
\IfFileExists{microtype.sty}{% use microtype if available
  \usepackage[]{microtype}
  \UseMicrotypeSet[protrusion]{basicmath} % disable protrusion for tt fonts
}{}
\makeatletter
\@ifundefined{KOMAClassName}{% if non-KOMA class
  \IfFileExists{parskip.sty}{%
    \usepackage{parskip}
  }{% else
    \setlength{\parindent}{0pt}
    \setlength{\parskip}{6pt plus 2pt minus 1pt}}
}{% if KOMA class
  \KOMAoptions{parskip=half}}
\makeatother
\usepackage{xcolor}
\IfFileExists{xurl.sty}{\usepackage{xurl}}{} % add URL line breaks if available
\IfFileExists{bookmark.sty}{\usepackage{bookmark}}{\usepackage{hyperref}}
\hypersetup{
  pdftitle={practical\_exercise\_2, Methods 3, 2021, autumn semester},
  pdfauthor={Caroline Hommel},
  hidelinks,
  pdfcreator={LaTeX via pandoc}}
\urlstyle{same} % disable monospaced font for URLs
\usepackage[margin=1in]{geometry}
\usepackage{color}
\usepackage{fancyvrb}
\newcommand{\VerbBar}{|}
\newcommand{\VERB}{\Verb[commandchars=\\\{\}]}
\DefineVerbatimEnvironment{Highlighting}{Verbatim}{commandchars=\\\{\}}
% Add ',fontsize=\small' for more characters per line
\usepackage{framed}
\definecolor{shadecolor}{RGB}{248,248,248}
\newenvironment{Shaded}{\begin{snugshade}}{\end{snugshade}}
\newcommand{\AlertTok}[1]{\textcolor[rgb]{0.94,0.16,0.16}{#1}}
\newcommand{\AnnotationTok}[1]{\textcolor[rgb]{0.56,0.35,0.01}{\textbf{\textit{#1}}}}
\newcommand{\AttributeTok}[1]{\textcolor[rgb]{0.77,0.63,0.00}{#1}}
\newcommand{\BaseNTok}[1]{\textcolor[rgb]{0.00,0.00,0.81}{#1}}
\newcommand{\BuiltInTok}[1]{#1}
\newcommand{\CharTok}[1]{\textcolor[rgb]{0.31,0.60,0.02}{#1}}
\newcommand{\CommentTok}[1]{\textcolor[rgb]{0.56,0.35,0.01}{\textit{#1}}}
\newcommand{\CommentVarTok}[1]{\textcolor[rgb]{0.56,0.35,0.01}{\textbf{\textit{#1}}}}
\newcommand{\ConstantTok}[1]{\textcolor[rgb]{0.00,0.00,0.00}{#1}}
\newcommand{\ControlFlowTok}[1]{\textcolor[rgb]{0.13,0.29,0.53}{\textbf{#1}}}
\newcommand{\DataTypeTok}[1]{\textcolor[rgb]{0.13,0.29,0.53}{#1}}
\newcommand{\DecValTok}[1]{\textcolor[rgb]{0.00,0.00,0.81}{#1}}
\newcommand{\DocumentationTok}[1]{\textcolor[rgb]{0.56,0.35,0.01}{\textbf{\textit{#1}}}}
\newcommand{\ErrorTok}[1]{\textcolor[rgb]{0.64,0.00,0.00}{\textbf{#1}}}
\newcommand{\ExtensionTok}[1]{#1}
\newcommand{\FloatTok}[1]{\textcolor[rgb]{0.00,0.00,0.81}{#1}}
\newcommand{\FunctionTok}[1]{\textcolor[rgb]{0.00,0.00,0.00}{#1}}
\newcommand{\ImportTok}[1]{#1}
\newcommand{\InformationTok}[1]{\textcolor[rgb]{0.56,0.35,0.01}{\textbf{\textit{#1}}}}
\newcommand{\KeywordTok}[1]{\textcolor[rgb]{0.13,0.29,0.53}{\textbf{#1}}}
\newcommand{\NormalTok}[1]{#1}
\newcommand{\OperatorTok}[1]{\textcolor[rgb]{0.81,0.36,0.00}{\textbf{#1}}}
\newcommand{\OtherTok}[1]{\textcolor[rgb]{0.56,0.35,0.01}{#1}}
\newcommand{\PreprocessorTok}[1]{\textcolor[rgb]{0.56,0.35,0.01}{\textit{#1}}}
\newcommand{\RegionMarkerTok}[1]{#1}
\newcommand{\SpecialCharTok}[1]{\textcolor[rgb]{0.00,0.00,0.00}{#1}}
\newcommand{\SpecialStringTok}[1]{\textcolor[rgb]{0.31,0.60,0.02}{#1}}
\newcommand{\StringTok}[1]{\textcolor[rgb]{0.31,0.60,0.02}{#1}}
\newcommand{\VariableTok}[1]{\textcolor[rgb]{0.00,0.00,0.00}{#1}}
\newcommand{\VerbatimStringTok}[1]{\textcolor[rgb]{0.31,0.60,0.02}{#1}}
\newcommand{\WarningTok}[1]{\textcolor[rgb]{0.56,0.35,0.01}{\textbf{\textit{#1}}}}
\usepackage{graphicx}
\makeatletter
\def\maxwidth{\ifdim\Gin@nat@width>\linewidth\linewidth\else\Gin@nat@width\fi}
\def\maxheight{\ifdim\Gin@nat@height>\textheight\textheight\else\Gin@nat@height\fi}
\makeatother
% Scale images if necessary, so that they will not overflow the page
% margins by default, and it is still possible to overwrite the defaults
% using explicit options in \includegraphics[width, height, ...]{}
\setkeys{Gin}{width=\maxwidth,height=\maxheight,keepaspectratio}
% Set default figure placement to htbp
\makeatletter
\def\fps@figure{htbp}
\makeatother
\setlength{\emergencystretch}{3em} % prevent overfull lines
\providecommand{\tightlist}{%
  \setlength{\itemsep}{0pt}\setlength{\parskip}{0pt}}
\setcounter{secnumdepth}{-\maxdimen} % remove section numbering
\ifLuaTeX
  \usepackage{selnolig}  % disable illegal ligatures
\fi

\begin{document}
\maketitle

\hypertarget{assignment-1-using-mixed-effects-modelling-to-model-hierarchical-data}{%
\section{Assignment 1: Using mixed effects modelling to model
hierarchical
data}\label{assignment-1-using-mixed-effects-modelling-to-model-hierarchical-data}}

In this assignment we will be investigating the \emph{politeness}
dataset of Winter and Grawunder (2012) and apply basic methods of
multilevel modelling.

\begin{Shaded}
\begin{Highlighting}[]
\NormalTok{politeness }\OtherTok{\textless{}{-}} \FunctionTok{read.csv}\NormalTok{(}\StringTok{\textquotesingle{}politeness.csv\textquotesingle{}}\NormalTok{) }\DocumentationTok{\#\# read in data}
\end{Highlighting}
\end{Shaded}

\hypertarget{exercise-1---describing-the-dataset-and-making-some-initial-plots}{%
\section{Exercise 1 - describing the dataset and making some initial
plots}\label{exercise-1---describing-the-dataset-and-making-some-initial-plots}}

\#\#1) Describe the dataset, such that someone who happened upon this
dataset could understand the variables and what they contain

\emph{Subject} The first letter indicates whether it's a male or female
(f or m) and the number indicates which number participant it is.

\emph{Gender} gender of the participant. F for female and m for male.

\emph{Scenario} The participants have been asked to behave within
different scenarios -7 different e.g.~asking for a favor or taking
orders.

\emph{Attitude} Whether it's formal or informal in a condition

\emph{Total duration} For how long the participants are talking

\emph{f0mm} Frequency in hertz

\emph{his\_count} The amount of noisy breath intakes during the
response.

\#\#\#i. Also consider whether any of the variables in \emph{politeness}
should be encoded as factors or have the factor encoding removed. Hint:
\texttt{?factor}

\begin{Shaded}
\begin{Highlighting}[]
\CommentTok{\#Changed all characters to factors in one line}
\NormalTok{politeness[}\FunctionTok{sapply}\NormalTok{(politeness, is.character)] }\OtherTok{\textless{}{-}} \FunctionTok{lapply}\NormalTok{(politeness[}\FunctionTok{sapply}\NormalTok{(politeness, is.character)], as.factor)}

\CommentTok{\#Changing scenario from integer to factor}
\NormalTok{politeness}\SpecialCharTok{$}\NormalTok{scenario }\OtherTok{\textless{}{-}} \FunctionTok{as.integer}\NormalTok{(politeness}\SpecialCharTok{$}\NormalTok{scenario)}
\end{Highlighting}
\end{Shaded}

\#\#2) Create a new data frame that just contains the subject \emph{F1}
and run two linear models; one that expresses \emph{f0mn} as dependent
on \emph{scenario} as an integer; and one that expresses \emph{f0mn} as
dependent on \emph{scenario} encoded as a factor

\begin{Shaded}
\begin{Highlighting}[]
\CommentTok{\#creating a data frame only with data from subject F1}
\NormalTok{f1data }\OtherTok{\textless{}{-}}\NormalTok{ politeness }\SpecialCharTok{\%\textgreater{}\%} 
  \FunctionTok{filter}\NormalTok{(subject }\SpecialCharTok{==} \StringTok{"F1"}\NormalTok{)}

\NormalTok{model }\OtherTok{\textless{}{-}} \FunctionTok{lm}\NormalTok{(f0mn}\SpecialCharTok{\textasciitilde{}}\NormalTok{scenario, }\AttributeTok{data =}\NormalTok{ f1data)}
\FunctionTok{summary}\NormalTok{(model)}
\end{Highlighting}
\end{Shaded}

\begin{verbatim}
## 
## Call:
## lm(formula = f0mn ~ scenario, data = f1data)
## 
## Residuals:
##     Min      1Q  Median      3Q     Max 
## -44.836 -36.807   6.686  20.918  46.421 
## 
## Coefficients:
##             Estimate Std. Error t value Pr(>|t|)    
## (Intercept)  262.621     20.616  12.738 2.48e-08 ***
## scenario      -6.886      4.610  -1.494    0.161    
## ---
## Signif. codes:  0 '***' 0.001 '**' 0.01 '*' 0.05 '.' 0.1 ' ' 1
## 
## Residual standard error: 34.5 on 12 degrees of freedom
## Multiple R-squared:  0.1568, Adjusted R-squared:  0.0865 
## F-statistic: 2.231 on 1 and 12 DF,  p-value: 0.1611
\end{verbatim}

\begin{Shaded}
\begin{Highlighting}[]
\NormalTok{f1data}\SpecialCharTok{$}\NormalTok{scenario }\OtherTok{\textless{}{-}} \FunctionTok{as.factor}\NormalTok{(f1data}\SpecialCharTok{$}\NormalTok{scenario)}
\NormalTok{model2 }\OtherTok{\textless{}{-}} \FunctionTok{lm}\NormalTok{(f0mn}\SpecialCharTok{\textasciitilde{}}\NormalTok{scenario, }\AttributeTok{data =}\NormalTok{ f1data)}
\FunctionTok{summary}\NormalTok{(model2)}
\end{Highlighting}
\end{Shaded}

\begin{verbatim}
## 
## Call:
## lm(formula = f0mn ~ scenario, data = f1data)
## 
## Residuals:
##    Min     1Q Median     3Q    Max 
## -37.50 -13.86   0.00  13.86  37.50 
## 
## Coefficients:
##             Estimate Std. Error t value Pr(>|t|)    
## (Intercept)   212.75      20.35  10.453  1.6e-05 ***
## scenario2      62.40      28.78   2.168   0.0668 .  
## scenario3      35.35      28.78   1.228   0.2591    
## scenario4      53.75      28.78   1.867   0.1041    
## scenario5      27.30      28.78   0.948   0.3745    
## scenario6      -7.55      28.78  -0.262   0.8006    
## scenario7     -14.95      28.78  -0.519   0.6195    
## ---
## Signif. codes:  0 '***' 0.001 '**' 0.01 '*' 0.05 '.' 0.1 ' ' 1
## 
## Residual standard error: 28.78 on 7 degrees of freedom
## Multiple R-squared:  0.6576, Adjusted R-squared:  0.364 
## F-statistic:  2.24 on 6 and 7 DF,  p-value: 0.1576
\end{verbatim}

\#\#\#i. Include the model matrices, \(X\) from the General Linear
Model, for these two models in your report and describe the different
interpretations of \emph{scenario} that these entail

\begin{Shaded}
\begin{Highlighting}[]
\CommentTok{\#creating design matrices }
\FunctionTok{model.matrix}\NormalTok{(model)}
\end{Highlighting}
\end{Shaded}

\begin{verbatim}
##    (Intercept) scenario
## 1            1        1
## 2            1        1
## 3            1        2
## 4            1        2
## 5            1        3
## 6            1        3
## 7            1        4
## 8            1        4
## 9            1        5
## 10           1        5
## 11           1        6
## 12           1        6
## 13           1        7
## 14           1        7
## attr(,"assign")
## [1] 0 1
\end{verbatim}

\begin{Shaded}
\begin{Highlighting}[]
\FunctionTok{model.matrix}\NormalTok{(model2)}
\end{Highlighting}
\end{Shaded}

\begin{verbatim}
##    (Intercept) scenario2 scenario3 scenario4 scenario5 scenario6 scenario7
## 1            1         0         0         0         0         0         0
## 2            1         0         0         0         0         0         0
## 3            1         1         0         0         0         0         0
## 4            1         1         0         0         0         0         0
## 5            1         0         1         0         0         0         0
## 6            1         0         1         0         0         0         0
## 7            1         0         0         1         0         0         0
## 8            1         0         0         1         0         0         0
## 9            1         0         0         0         1         0         0
## 10           1         0         0         0         1         0         0
## 11           1         0         0         0         0         1         0
## 12           1         0         0         0         0         1         0
## 13           1         0         0         0         0         0         1
## 14           1         0         0         0         0         0         1
## attr(,"assign")
## [1] 0 1 1 1 1 1 1
## attr(,"contrasts")
## attr(,"contrasts")$scenario
## [1] "contr.treatment"
\end{verbatim}

\#\#\#ii. Which coding of \emph{scenario}, as a factor or not, is more
fitting? Since the scenarios are nominal and cannot be `compared' to
each other, since e.g.~scenario 1 is not double as much anything as
scenario 2. We need to create two matrices, because the first one
doesn't show that all scenarios need to `weigh' as much as each other.

\#\#3) Make a plot that includes a subplot for each subject that has
\emph{scenario} on the x-axis and \emph{f0mn} on the y-axis and where
points are colour coded according to \emph{attitude}

\begin{Shaded}
\begin{Highlighting}[]
\CommentTok{\#making a plot of all the subjects }
\FunctionTok{ggplot}\NormalTok{(politeness, }\FunctionTok{aes}\NormalTok{(}\AttributeTok{x=}\NormalTok{scenario, }\AttributeTok{y=}\NormalTok{f0mn, }\AttributeTok{color =}\NormalTok{ attitude))}\SpecialCharTok{+}
  \FunctionTok{geom\_point}\NormalTok{()}\SpecialCharTok{+}
\CommentTok{\#like random intercept {-}each subject will get their own little plot}
  \FunctionTok{facet\_wrap}\NormalTok{(}\SpecialCharTok{\textasciitilde{}}\NormalTok{subject)}\SpecialCharTok{+}
  \FunctionTok{theme\_minimal}\NormalTok{()}\SpecialCharTok{+}
  \FunctionTok{ylab}\NormalTok{(}\StringTok{"Frequency in Hz"}\NormalTok{)}\SpecialCharTok{+}
  \FunctionTok{xlab}\NormalTok{(}\StringTok{"Scenario"}\NormalTok{)}
\end{Highlighting}
\end{Shaded}

\begin{verbatim}
## Warning: Removed 12 rows containing missing values (geom_point).
\end{verbatim}

\includegraphics{my-own-practical_exercise_2-kopi_files/figure-latex/unnamed-chunk-5-1.pdf}

\#\#\#i. Describe the differences between subjects All the M's have
lower frequency, which makes sense since it's the plot of the mens
voices.It looks like the effect of attitude is larger on some subjects
than other.For the men their is almost no difference between informal
and polite.

\hypertarget{exercise-2---comparison-of-models}{%
\section{Exercise 2 - comparison of
models}\label{exercise-2---comparison-of-models}}

\begin{Shaded}
\begin{Highlighting}[]
\NormalTok{example.formula }\OtherTok{\textless{}{-}} \FunctionTok{formula}\NormalTok{(dep.variable }\SpecialCharTok{\textasciitilde{}}\NormalTok{ first.level.variable }\SpecialCharTok{+}\NormalTok{ (}\DecValTok{1} \SpecialCharTok{|}\NormalTok{ second.level.variable))}
\end{Highlighting}
\end{Shaded}

\#\#1) Build four models and do some comparisons i. a single level model
that models \emph{f0mn} as dependent on \emph{gender}

\begin{Shaded}
\begin{Highlighting}[]
\NormalTok{single }\OtherTok{\textless{}{-}} \FunctionTok{lm}\NormalTok{(f0mn }\SpecialCharTok{\textasciitilde{}}\NormalTok{gender, }\AttributeTok{data =}\NormalTok{ politeness)}
\end{Highlighting}
\end{Shaded}

\#\#\#ii. a two-level model that adds a second level on top of i. where
unique intercepts are modelled for each \emph{scenario}

\begin{Shaded}
\begin{Highlighting}[]
\NormalTok{twolevel }\OtherTok{\textless{}{-}} \FunctionTok{lmer}\NormalTok{(f0mn }\SpecialCharTok{\textasciitilde{}}\NormalTok{ gender }\SpecialCharTok{+}\NormalTok{ (}\DecValTok{1} \SpecialCharTok{|}\NormalTok{ scenario), }\AttributeTok{data =}\NormalTok{ politeness)}
\end{Highlighting}
\end{Shaded}

\#\#\#iii. a two-level model that only has \emph{subject} as an
intercept

\begin{Shaded}
\begin{Highlighting}[]
\NormalTok{twolevel2 }\OtherTok{\textless{}{-}} \FunctionTok{lmer}\NormalTok{(f0mn }\SpecialCharTok{\textasciitilde{}}\NormalTok{ gender }\SpecialCharTok{+}\NormalTok{ (}\DecValTok{1} \SpecialCharTok{|}\NormalTok{ subject), }\AttributeTok{data =}\NormalTok{ politeness)}
\end{Highlighting}
\end{Shaded}

\#\#\#iv. a two-level model that models intercepts for both
\emph{scenario} and \emph{subject}

\begin{Shaded}
\begin{Highlighting}[]
\NormalTok{twolevel3 }\OtherTok{\textless{}{-}} \FunctionTok{lmer}\NormalTok{(f0mn }\SpecialCharTok{\textasciitilde{}}\NormalTok{ gender }\SpecialCharTok{+}\NormalTok{ (}\DecValTok{1} \SpecialCharTok{|}\NormalTok{ subject) }\SpecialCharTok{+}\NormalTok{ (}\DecValTok{1} \SpecialCharTok{|}\NormalTok{ scenario), }\AttributeTok{data =}\NormalTok{ politeness)}
\end{Highlighting}
\end{Shaded}

\#\#\#v. which of the models has the lowest residual standard deviation,
also compare the Akaike Information Criterion \texttt{AIC}?

\begin{Shaded}
\begin{Highlighting}[]
\CommentTok{\#calculate the residual standard deviation....}

\FunctionTok{AIC}\NormalTok{(single, twolevel, twolevel2, twolevel3)}
\end{Highlighting}
\end{Shaded}

\begin{verbatim}
##           df      AIC
## single     3 2163.971
## twolevel   4 2152.314
## twolevel2  4 2099.626
## twolevel3  5 2092.482
\end{verbatim}

\begin{Shaded}
\begin{Highlighting}[]
\NormalTok{tabel }\OtherTok{\textless{}{-}} \FunctionTok{c}\NormalTok{(}\StringTok{\textquotesingle{}single\textquotesingle{}}\NormalTok{, }\StringTok{\textquotesingle{}twolevel\textquotesingle{}}\NormalTok{, }\StringTok{\textquotesingle{}twolevel2\textquotesingle{}}\NormalTok{, }\StringTok{\textquotesingle{}twolevel3\textquotesingle{}}\NormalTok{)}

\CommentTok{\#from lme4 {-}extracting residual standard deviation }
\NormalTok{sigma }\OtherTok{\textless{}{-}} \FunctionTok{c}\NormalTok{(}\FunctionTok{sigma}\NormalTok{(single), }\FunctionTok{sigma}\NormalTok{(twolevel), }\FunctionTok{sigma}\NormalTok{(twolevel2), }\FunctionTok{sigma}\NormalTok{(twolevel3))}

\FunctionTok{as\_tibble}\NormalTok{(}\FunctionTok{rbind}\NormalTok{(tabel, sigma))}
\end{Highlighting}
\end{Shaded}

\begin{verbatim}
## Warning: The `x` argument of `as_tibble.matrix()` must have unique column names if `.name_repair` is omitted as of tibble 2.0.0.
## Using compatibility `.name_repair`.
## This warning is displayed once every 8 hours.
## Call `lifecycle::last_warnings()` to see where this warning was generated.
\end{verbatim}

\begin{verbatim}
## # A tibble: 2 x 4
##   V1               V2               V3               V4              
##   <chr>            <chr>            <chr>            <chr>           
## 1 single           twolevel         twolevel2        twolevel3       
## 2 39.4626791688879 38.4480006729009 32.0428716618224 30.6580347795662
\end{verbatim}

\begin{Shaded}
\begin{Highlighting}[]
\CommentTok{\#(or data.frame(rbind(tabel, sigma))}
\end{Highlighting}
\end{Shaded}

The two-level model that models intercepts for both \emph{scenario} and
\emph{subject} has the lowest AIC (2092) and the lowest residual
standard deviation (30.65).

\#\#\#vi. which of the second-level effects explains the most variance?
The model with \emph{subject} as the second level effect (random
intercept) explains the variance the better than the model with
\emph{scenario} as the second level effect.

\#\#2) Why is our single-level model bad? The model doesn't take
individual differences in pitch into account.

\#\#\#i. create a new data frame that has three variables,
\emph{subject}, \emph{gender} and \emph{f0mn}, where \emph{f0mn} is the
average of all responses of each subject, i.e.~averaging across
\emph{attitude} and \emph{scenario}

\begin{Shaded}
\begin{Highlighting}[]
\CommentTok{\#na.omit removes all na\textquotesingle{}values\textquotesingle{} from the data set }
\NormalTok{new\_df }\OtherTok{\textless{}{-}}\NormalTok{ politeness }\SpecialCharTok{\%\textgreater{}\%} 
\NormalTok{  na.omit }\SpecialCharTok{\%\textgreater{}\%} 
  \FunctionTok{group\_by}\NormalTok{(subject, gender) }\SpecialCharTok{\%\textgreater{}\%} 
  \FunctionTok{summarise}\NormalTok{(}\AttributeTok{avg\_f0mn=}\NormalTok{(}\FunctionTok{mean}\NormalTok{(f0mn)))}
\end{Highlighting}
\end{Shaded}

\begin{verbatim}
## `summarise()` has grouped output by 'subject'. You can override using the `.groups` argument.
\end{verbatim}

\#\#\#ii. build a single-level model that models \emph{f0mn} as
dependent on \emph{gender} using this new data set

\begin{Shaded}
\begin{Highlighting}[]
\NormalTok{new\_single }\OtherTok{\textless{}{-}} \FunctionTok{lm}\NormalTok{(avg\_f0mn}\SpecialCharTok{\textasciitilde{}}\NormalTok{gender, }\AttributeTok{data=}\NormalTok{new\_df)}
\end{Highlighting}
\end{Shaded}

\#\#\#iii. make Quantile-Quantile plots, comparing theoretical quantiles
to the sample quantiles using \texttt{qqnorm} and \texttt{qqline} for
the new single-level model and compare it to the old single-level model
(from 1).i).

\begin{Shaded}
\begin{Highlighting}[]
\FunctionTok{qqnorm}\NormalTok{(}\FunctionTok{resid}\NormalTok{(single), }\AttributeTok{pch =} \DecValTok{1}\NormalTok{, }\AttributeTok{frame =} \ConstantTok{FALSE}\NormalTok{)}
\FunctionTok{qqline}\NormalTok{(}\FunctionTok{resid}\NormalTok{(single), }\AttributeTok{col =} \StringTok{"steelblue"}\NormalTok{, }\AttributeTok{lwd =} \DecValTok{2}\NormalTok{)}
\end{Highlighting}
\end{Shaded}

\includegraphics{my-own-practical_exercise_2-kopi_files/figure-latex/unnamed-chunk-14-1.pdf}

\begin{Shaded}
\begin{Highlighting}[]
\FunctionTok{qqnorm}\NormalTok{(}\FunctionTok{resid}\NormalTok{(new\_single),}\AttributeTok{pch =} \DecValTok{1}\NormalTok{, }\AttributeTok{frame =} \ConstantTok{FALSE}\NormalTok{)}
\FunctionTok{qqline}\NormalTok{(}\FunctionTok{resid}\NormalTok{(new\_single), }\AttributeTok{col =} \StringTok{"steelblue"}\NormalTok{, }\AttributeTok{lwd =} \DecValTok{2}\NormalTok{)}
\end{Highlighting}
\end{Shaded}

\includegraphics{my-own-practical_exercise_2-kopi_files/figure-latex/unnamed-chunk-14-2.pdf}
Which model's residuals (\(\epsilon\)) fulfill the assumptions of the
General Linear Model better? It looks like plot number two with the
averaged frequency pr. participant (with the fewest data points) is more
linear and the data points look more equally distributed on each side of
the blue line. The first model looks more systematically distributed and
the data points are not as equally distributed on each sides of the blue
line.

\#\#\#iv. Also make a quantile-quantile plot for the residuals of the
multilevel model with two intercepts. Does it look alright?

\begin{Shaded}
\begin{Highlighting}[]
\FunctionTok{qqnorm}\NormalTok{(}\FunctionTok{resid}\NormalTok{(twolevel3),}\AttributeTok{pch =} \DecValTok{1}\NormalTok{, }\AttributeTok{frame =} \ConstantTok{FALSE}\NormalTok{)}
\FunctionTok{qqline}\NormalTok{(}\FunctionTok{resid}\NormalTok{(twolevel3), }\AttributeTok{col =} \StringTok{"steelblue"}\NormalTok{, }\AttributeTok{lwd =} \DecValTok{2}\NormalTok{)}
\end{Highlighting}
\end{Shaded}

\includegraphics{my-own-practical_exercise_2-kopi_files/figure-latex/unnamed-chunk-15-1.pdf}
It is a little better than the others, since there is still some
deviation from the line in the outer quantiles, but all along the middle
the data points follows the blue line quite well.

\#\#3) Plotting the two-intercepts model \#\#\#i. Create a plot for each
subject, (similar to part 3 in Exercise 1), this time also indicating
the fitted value for each of the subjects for each for the scenarios
(hint use \texttt{fixef} to get the ``grand effects'' for each gender
and \texttt{ranef} to get the subject- and scenario-specific effects)

\begin{Shaded}
\begin{Highlighting}[]
\FunctionTok{library}\NormalTok{(}\StringTok{"lme4"}\NormalTok{)}
\FunctionTok{data}\NormalTok{(}\AttributeTok{package =} \StringTok{"lme4"}\NormalTok{)}

\NormalTok{politeness }\OtherTok{\textless{}{-}}\NormalTok{ politeness }\SpecialCharTok{\%\textgreater{}\%} \FunctionTok{na.omit}\NormalTok{(politeness)}
\CommentTok{\#new column with the predicted y{-}values }
\NormalTok{politeness}\SpecialCharTok{$}\NormalTok{yhat }\OtherTok{\textless{}{-}} \FunctionTok{predict}\NormalTok{(twolevel3)}

\FunctionTok{ggplot}\NormalTok{(politeness, }\FunctionTok{aes}\NormalTok{(}\AttributeTok{x =}\NormalTok{ scenario, }\AttributeTok{y =}\NormalTok{ f0mn))}\SpecialCharTok{+}
  \FunctionTok{geom\_point}\NormalTok{() }\SpecialCharTok{+} 
  \FunctionTok{geom\_point}\NormalTok{(}\FunctionTok{aes}\NormalTok{(}\AttributeTok{x =}\NormalTok{ scenario, }\AttributeTok{y =}\NormalTok{ yhat), }\AttributeTok{col =} \StringTok{"blue"}\NormalTok{, }\AttributeTok{shape =} \DecValTok{3}\NormalTok{) }\SpecialCharTok{+} 
  \FunctionTok{facet\_wrap}\NormalTok{(}\SpecialCharTok{\textasciitilde{}}\NormalTok{ subject)}\SpecialCharTok{+} 
  \FunctionTok{xlab}\NormalTok{(}\StringTok{\textquotesingle{}Scenario\textquotesingle{}}\NormalTok{) }\SpecialCharTok{+}
  \FunctionTok{ylab}\NormalTok{(}\StringTok{\textquotesingle{}Hz\textquotesingle{}}\NormalTok{) }\SpecialCharTok{+}
  \FunctionTok{theme\_minimal}\NormalTok{()}
\end{Highlighting}
\end{Shaded}

\includegraphics{my-own-practical_exercise_2-kopi_files/figure-latex/unnamed-chunk-16-1.pdf}
Blue = predicted and black = the actual observed values.

\hypertarget{exercise-3---now-with-attitude}{%
\section{Exercise 3 - now with
attitude}\label{exercise-3---now-with-attitude}}

\#\#1) Carry on with the model with the two unique intercepts fitted (
\emph{scenario} and \emph{subject}). i. now build a model that has
\emph{attitude} as a main effect besides \emph{gender}

\begin{Shaded}
\begin{Highlighting}[]
\NormalTok{doubble\_model }\OtherTok{\textless{}{-}} \FunctionTok{lmer}\NormalTok{(f0mn }\SpecialCharTok{\textasciitilde{}}\NormalTok{ gender }\SpecialCharTok{+}\NormalTok{ attitude }\SpecialCharTok{+}\NormalTok{ (}\DecValTok{1} \SpecialCharTok{|}\NormalTok{ subject) }\SpecialCharTok{+}\NormalTok{ (}\DecValTok{1} \SpecialCharTok{|}\NormalTok{ scenario), }\AttributeTok{data =}\NormalTok{ politeness)}
\end{Highlighting}
\end{Shaded}

\#\#\#ii. make a separate model that besides the main effects of
\emph{attitude} and \emph{gender} also include their interaction

\begin{Shaded}
\begin{Highlighting}[]
\NormalTok{interaction\_model }\OtherTok{\textless{}{-}} \FunctionTok{lmer}\NormalTok{(f0mn }\SpecialCharTok{\textasciitilde{}}\NormalTok{ attitude}\SpecialCharTok{*}\NormalTok{gender }\SpecialCharTok{+}\NormalTok{ (}\DecValTok{1} \SpecialCharTok{|}\NormalTok{ subject) }\SpecialCharTok{+}\NormalTok{ (}\DecValTok{1} \SpecialCharTok{|}\NormalTok{ scenario), }\AttributeTok{data =}\NormalTok{ politeness)}
\end{Highlighting}
\end{Shaded}

\#\#\#iii. describe what the interaction term in the model says about
Korean males pitch when they are polite relative to Korean women's pitch
when they are polite (you don't have to judge whether it is interesting)

How much the effects of attitude polite changes when it goes from female
to male woman's frequency is influenced more by the change of attitude
than male's frequency. But since the error bars overlap the mean
frequency could possible lay within this error bar and therefore there
is a uncertainty according to if there is an effect.

\begin{Shaded}
\begin{Highlighting}[]
\FunctionTok{plot}\NormalTok{(effects}\SpecialCharTok{::}\FunctionTok{allEffects}\NormalTok{(interaction\_model))}
\end{Highlighting}
\end{Shaded}

\includegraphics{my-own-practical_exercise_2-kopi_files/figure-latex/unnamed-chunk-19-1.pdf}

\begin{Shaded}
\begin{Highlighting}[]
\FunctionTok{plot}\NormalTok{(}\FunctionTok{allEffects}\NormalTok{(interaction\_model), }\AttributeTok{multiline=}\ConstantTok{TRUE}\NormalTok{, }\AttributeTok{ci.style=}\StringTok{"bars"}\NormalTok{)}
\end{Highlighting}
\end{Shaded}

\includegraphics{my-own-practical_exercise_2-kopi_files/figure-latex/unnamed-chunk-19-2.pdf}

\#\#2) Compare the three models (1. gender as a main effect; 2. gender
and attitude as main effects; 3. gender and attitude as main effects and
the interaction between them. For all three models model unique
intercepts for \emph{subject} and \emph{scenario}) using residual
variance, residual standard deviation and AIC.

\begin{Shaded}
\begin{Highlighting}[]
\NormalTok{AIC }\OtherTok{\textless{}{-}}\FunctionTok{AIC}\NormalTok{(twolevel3, doubble\_model, interaction\_model)}

\CommentTok{\#to be able to add the names into a scheme }
\NormalTok{tabel2 }\OtherTok{\textless{}{-}} \FunctionTok{c}\NormalTok{(}\StringTok{\textquotesingle{}twolevel3\textquotesingle{}}\NormalTok{, }\StringTok{\textquotesingle{}doubble\_model\textquotesingle{}}\NormalTok{, }\StringTok{\textquotesingle{}interaction\_model\textquotesingle{}}\NormalTok{)}

\CommentTok{\#from lme4 {-}extracting residual standard deviation }
\NormalTok{sigma2 }\OtherTok{\textless{}{-}} \FunctionTok{c}\NormalTok{(}\FunctionTok{sigma}\NormalTok{(twolevel3), }\FunctionTok{sigma}\NormalTok{(doubble\_model), }\FunctionTok{sigma}\NormalTok{(interaction\_model))}

\CommentTok{\#residual variance}
\NormalTok{variances }\OtherTok{\textless{}{-}} \FunctionTok{c}\NormalTok{(}\FunctionTok{var}\NormalTok{(}\FunctionTok{resid}\NormalTok{(twolevel3)), }\FunctionTok{var}\NormalTok{(}\FunctionTok{resid}\NormalTok{(doubble\_model)),}\FunctionTok{var}\NormalTok{(}\FunctionTok{resid}\NormalTok{(interaction\_model)))}

\CommentTok{\#collecting them all in one scheme }
\FunctionTok{as.tibble}\NormalTok{(}\FunctionTok{cbind}\NormalTok{(tabel2, sigma2, AIC, variances))}
\end{Highlighting}
\end{Shaded}

\begin{verbatim}
## Warning: `as.tibble()` was deprecated in tibble 2.0.0.
## Please use `as_tibble()` instead.
## The signature and semantics have changed, see `?as_tibble`.
## This warning is displayed once every 8 hours.
## Call `lifecycle::last_warnings()` to see where this warning was generated.
\end{verbatim}

\begin{verbatim}
## # A tibble: 3 x 5
##   tabel2            sigma2    df   AIC variances
##   <chr>              <dbl> <dbl> <dbl>     <dbl>
## 1 twolevel3           30.7     5 2092.      860.
## 2 doubble_model       29.7     6 2077.      802.
## 3 interaction_model   29.8     7 2073.      800.
\end{verbatim}

The model with gender as the main effect has the highest value in all
three methods of comparison.

\#\#3) Choose the model that you think describe the data the best - and
write a short report on the main findings based on this model.

The data set consists of frequency measurements of 7 men and 9 women.
Their voice frequency was measured over 7 different scenarios, in two
different attitudes - polite and informal. This means all participants
went through 14 trials in total (except for a few missing values). Other
variables include gender, hiss count and total duration of the specific
trial.

After running different models on the data, the following model was
chosen to investigate the effect of gender and attitude on pitch.

frequency \textasciitilde{} gender + attitude + (1 \textbar{} subject) +
(1 \textbar{} scenario)

The dependent variable is the pitch frequency, the fixed effects include
gender and attitude and subject and scenario are modelled as random
intercepts. Since humans naturally have different pitch frequencies, and
what we are interested in is how it is changed under certain
circumstances, it is relevant to include random intercepts for each
subject. If it was not included, the model would not take into account
the repeated measurement design of the experiment. Furthermore, if you
do not include random intercepts, you would overlook a very clear
underlying effect in the data, thus having the risk of not interpreting
the model properly. Therefore, including random intercepts for subject
and scenario was found important.

This model was chosen since compared to a model with only gender as a
fixed effect (and random intercepts for subject and scenario), the
chosen model explained more variance. A more complex model including the
interaction between gender and attitude could have been chosen, but
using the \emph{anova} function we found the interaction to be
insignificant. Thus a simpler model was preferred.

After building the model a quantile-quantile plot of the chosen model
was made testing for the assumption of normality of the residuals.

\begin{Shaded}
\begin{Highlighting}[]
\FunctionTok{qqnorm}\NormalTok{(}\FunctionTok{resid}\NormalTok{(doubble\_model))}
\FunctionTok{qqline}\NormalTok{(}\FunctionTok{resid}\NormalTok{(doubble\_model))}
\end{Highlighting}
\end{Shaded}

\includegraphics{my-own-practical_exercise_2-kopi_files/figure-latex/unnamed-chunk-21-1.pdf}
Figure 1. Quantile Quantile plot for the residuals of the multilevel
model.

By checking the plot, it was concluded that the model fulfilled the
assumption, since most of the points are on the straight line, and we
only see a small pattern of deviation at the end of the line.

Investigating the coefficients, it was found that our intercept was
254.4, showing the average of the women's pitch frequency in the
informal condition. Gender significantly predicts frequency (β =
-115.14, p\textless{} .001). Additionally, it was found that attitude
significantly predicts frequency (β = -14.82, p\textless{} .001). This
means that males generally have a lower frequency compared to women, and
that changing attitude from informal to polite tends to result in a
lower frequency when the other variables are held constant.

We see that there is a higher variance (585.6) and std. dev. (24.20) for
the second level effect subject compared to the second level effect of
scenario where variance (106.7) and std. dev. (10.33) which means that
there is a higher variability within \emph{subjects} compared to
\emph{scenarios}.

The special thing shown by our model is that both Korean men and womens
frequency gets lower in polite scenarios compared to informal scenarios,
whereas in many other languages it is the opposite, when in polite
scenarios the pitch gets higher for both men and women (Winter, 2013)
ergo there are also cultural differences.

This is written in collaboration with all of study group four.

\end{document}
