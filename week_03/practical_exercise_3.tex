% Options for packages loaded elsewhere
\PassOptionsToPackage{unicode}{hyperref}
\PassOptionsToPackage{hyphens}{url}
%
\documentclass[
]{article}
\title{practical\_exercise\_3, Methods 3, 2021, autumn semester}
\author{Caroline Hommel}
\date{4th of October 2021}

\usepackage{amsmath,amssymb}
\usepackage{lmodern}
\usepackage{iftex}
\ifPDFTeX
  \usepackage[T1]{fontenc}
  \usepackage[utf8]{inputenc}
  \usepackage{textcomp} % provide euro and other symbols
\else % if luatex or xetex
  \usepackage{unicode-math}
  \defaultfontfeatures{Scale=MatchLowercase}
  \defaultfontfeatures[\rmfamily]{Ligatures=TeX,Scale=1}
\fi
% Use upquote if available, for straight quotes in verbatim environments
\IfFileExists{upquote.sty}{\usepackage{upquote}}{}
\IfFileExists{microtype.sty}{% use microtype if available
  \usepackage[]{microtype}
  \UseMicrotypeSet[protrusion]{basicmath} % disable protrusion for tt fonts
}{}
\makeatletter
\@ifundefined{KOMAClassName}{% if non-KOMA class
  \IfFileExists{parskip.sty}{%
    \usepackage{parskip}
  }{% else
    \setlength{\parindent}{0pt}
    \setlength{\parskip}{6pt plus 2pt minus 1pt}}
}{% if KOMA class
  \KOMAoptions{parskip=half}}
\makeatother
\usepackage{xcolor}
\IfFileExists{xurl.sty}{\usepackage{xurl}}{} % add URL line breaks if available
\IfFileExists{bookmark.sty}{\usepackage{bookmark}}{\usepackage{hyperref}}
\hypersetup{
  pdftitle={practical\_exercise\_3, Methods 3, 2021, autumn semester},
  pdfauthor={Caroline Hommel},
  hidelinks,
  pdfcreator={LaTeX via pandoc}}
\urlstyle{same} % disable monospaced font for URLs
\usepackage[margin=1in]{geometry}
\usepackage{color}
\usepackage{fancyvrb}
\newcommand{\VerbBar}{|}
\newcommand{\VERB}{\Verb[commandchars=\\\{\}]}
\DefineVerbatimEnvironment{Highlighting}{Verbatim}{commandchars=\\\{\}}
% Add ',fontsize=\small' for more characters per line
\usepackage{framed}
\definecolor{shadecolor}{RGB}{248,248,248}
\newenvironment{Shaded}{\begin{snugshade}}{\end{snugshade}}
\newcommand{\AlertTok}[1]{\textcolor[rgb]{0.94,0.16,0.16}{#1}}
\newcommand{\AnnotationTok}[1]{\textcolor[rgb]{0.56,0.35,0.01}{\textbf{\textit{#1}}}}
\newcommand{\AttributeTok}[1]{\textcolor[rgb]{0.77,0.63,0.00}{#1}}
\newcommand{\BaseNTok}[1]{\textcolor[rgb]{0.00,0.00,0.81}{#1}}
\newcommand{\BuiltInTok}[1]{#1}
\newcommand{\CharTok}[1]{\textcolor[rgb]{0.31,0.60,0.02}{#1}}
\newcommand{\CommentTok}[1]{\textcolor[rgb]{0.56,0.35,0.01}{\textit{#1}}}
\newcommand{\CommentVarTok}[1]{\textcolor[rgb]{0.56,0.35,0.01}{\textbf{\textit{#1}}}}
\newcommand{\ConstantTok}[1]{\textcolor[rgb]{0.00,0.00,0.00}{#1}}
\newcommand{\ControlFlowTok}[1]{\textcolor[rgb]{0.13,0.29,0.53}{\textbf{#1}}}
\newcommand{\DataTypeTok}[1]{\textcolor[rgb]{0.13,0.29,0.53}{#1}}
\newcommand{\DecValTok}[1]{\textcolor[rgb]{0.00,0.00,0.81}{#1}}
\newcommand{\DocumentationTok}[1]{\textcolor[rgb]{0.56,0.35,0.01}{\textbf{\textit{#1}}}}
\newcommand{\ErrorTok}[1]{\textcolor[rgb]{0.64,0.00,0.00}{\textbf{#1}}}
\newcommand{\ExtensionTok}[1]{#1}
\newcommand{\FloatTok}[1]{\textcolor[rgb]{0.00,0.00,0.81}{#1}}
\newcommand{\FunctionTok}[1]{\textcolor[rgb]{0.00,0.00,0.00}{#1}}
\newcommand{\ImportTok}[1]{#1}
\newcommand{\InformationTok}[1]{\textcolor[rgb]{0.56,0.35,0.01}{\textbf{\textit{#1}}}}
\newcommand{\KeywordTok}[1]{\textcolor[rgb]{0.13,0.29,0.53}{\textbf{#1}}}
\newcommand{\NormalTok}[1]{#1}
\newcommand{\OperatorTok}[1]{\textcolor[rgb]{0.81,0.36,0.00}{\textbf{#1}}}
\newcommand{\OtherTok}[1]{\textcolor[rgb]{0.56,0.35,0.01}{#1}}
\newcommand{\PreprocessorTok}[1]{\textcolor[rgb]{0.56,0.35,0.01}{\textit{#1}}}
\newcommand{\RegionMarkerTok}[1]{#1}
\newcommand{\SpecialCharTok}[1]{\textcolor[rgb]{0.00,0.00,0.00}{#1}}
\newcommand{\SpecialStringTok}[1]{\textcolor[rgb]{0.31,0.60,0.02}{#1}}
\newcommand{\StringTok}[1]{\textcolor[rgb]{0.31,0.60,0.02}{#1}}
\newcommand{\VariableTok}[1]{\textcolor[rgb]{0.00,0.00,0.00}{#1}}
\newcommand{\VerbatimStringTok}[1]{\textcolor[rgb]{0.31,0.60,0.02}{#1}}
\newcommand{\WarningTok}[1]{\textcolor[rgb]{0.56,0.35,0.01}{\textbf{\textit{#1}}}}
\usepackage{graphicx}
\makeatletter
\def\maxwidth{\ifdim\Gin@nat@width>\linewidth\linewidth\else\Gin@nat@width\fi}
\def\maxheight{\ifdim\Gin@nat@height>\textheight\textheight\else\Gin@nat@height\fi}
\makeatother
% Scale images if necessary, so that they will not overflow the page
% margins by default, and it is still possible to overwrite the defaults
% using explicit options in \includegraphics[width, height, ...]{}
\setkeys{Gin}{width=\maxwidth,height=\maxheight,keepaspectratio}
% Set default figure placement to htbp
\makeatletter
\def\fps@figure{htbp}
\makeatother
\setlength{\emergencystretch}{3em} % prevent overfull lines
\providecommand{\tightlist}{%
  \setlength{\itemsep}{0pt}\setlength{\parskip}{0pt}}
\setcounter{secnumdepth}{-\maxdimen} % remove section numbering
\ifLuaTeX
  \usepackage{selnolig}  % disable illegal ligatures
\fi

\begin{document}
\maketitle

\hypertarget{exercises-and-objectives}{%
\section{Exercises and objectives}\label{exercises-and-objectives}}

The objectives of the exercises of this assignment are:\\
1) Download and organised the data and model and plot staircase
responses based on fits of logistic functions\\
2) Fit multilevel models for response times\\
3) Fit multilevel models for count data

\hypertarget{exercise-1}{%
\subsection{Exercise 1}\label{exercise-1}}

\begin{enumerate}
\def\labelenumi{\arabic{enumi})}
\tightlist
\item
  Put the data from all subjects into a single data frame
\end{enumerate}

\begin{Shaded}
\begin{Highlighting}[]
\CommentTok{\# Making a data frame with the data from experiment 2}
\NormalTok{lau\_files }\OtherTok{\textless{}{-}} \FunctionTok{list.files}\NormalTok{(}\AttributeTok{path =} \StringTok{"experiment\_2"}\NormalTok{,}
                    \AttributeTok{pattern =} \StringTok{".csv"}\NormalTok{,}
                    \AttributeTok{full.names =}\NormalTok{ T)}

\NormalTok{lau }\OtherTok{\textless{}{-}} \FunctionTok{data.frame}\NormalTok{() }\CommentTok{\# Creating empty data frame}

\ControlFlowTok{for}\NormalTok{ (i }\ControlFlowTok{in} \DecValTok{1}\SpecialCharTok{:}\FunctionTok{length}\NormalTok{(lau\_files))\{}
\NormalTok{  new\_dat }\OtherTok{\textless{}{-}} \FunctionTok{read.table}\NormalTok{(lau\_files[i], }\AttributeTok{sep =} \StringTok{","}\NormalTok{, }\AttributeTok{header =} \ConstantTok{TRUE}\NormalTok{, }\AttributeTok{dec =} \StringTok{\textquotesingle{},\textquotesingle{}}\NormalTok{ )}
\NormalTok{  lau }\OtherTok{\textless{}{-}} \FunctionTok{rbind}\NormalTok{(new\_dat, lau)}
\NormalTok{\} }\CommentTok{\# Going through each of the files on the list, reading them and them adding them to the data frame}
\end{Highlighting}
\end{Shaded}

\begin{enumerate}
\def\labelenumi{\arabic{enumi})}
\setcounter{enumi}{1}
\tightlist
\item
  Describe the data and construct extra variables from the existing
  variables

  \begin{enumerate}
  \def\labelenumii{\roman{enumii}.}
  \tightlist
  \item
    add a variable to the data frame and call it \emph{correct} (have it
    be a \emph{logical} variable). Assign a 1 to each row where the
    subject indicated the correct answer and a 0 to each row where the
    subject indicated the incorrect answer (\textbf{Hint:} the variable
    \emph{obj.resp} indicates whether the subject answered ``even'',
    \emph{e} or ``odd'', \emph{o}, and the variable \emph{target\_type}
    indicates what was actually presented.
  \end{enumerate}
\end{enumerate}

\begin{Shaded}
\begin{Highlighting}[]
\CommentTok{\#Changing even and odd to e\textquotesingle{}s and o\textquotesingle{}s}
\NormalTok{lau}\SpecialCharTok{$}\NormalTok{target.type }\OtherTok{\textless{}{-}} \FunctionTok{replace}\NormalTok{(lau}\SpecialCharTok{$}\NormalTok{target.type, lau}\SpecialCharTok{$}\NormalTok{target.type }\SpecialCharTok{==} \StringTok{"even"}\NormalTok{, }\StringTok{"e"}\NormalTok{)}
\NormalTok{lau}\SpecialCharTok{$}\NormalTok{target.type }\OtherTok{\textless{}{-}} \FunctionTok{replace}\NormalTok{(lau}\SpecialCharTok{$}\NormalTok{target.type, lau}\SpecialCharTok{$}\NormalTok{target.type }\SpecialCharTok{==} \StringTok{"odd"}\NormalTok{, }\StringTok{"o"}\NormalTok{)}

\CommentTok{\#Creating a new column called \textquotesingle{}correct\textquotesingle{} and telling it to write 1 if the same letter is present in target.type and obj.resp and 0 if it is not correct }

\NormalTok{lau}\SpecialCharTok{$}\NormalTok{correct }\OtherTok{\textless{}{-}} \FunctionTok{ifelse}\NormalTok{((lau}\SpecialCharTok{$}\NormalTok{target.type }\SpecialCharTok{==}\NormalTok{ lau}\SpecialCharTok{$}\NormalTok{obj.resp), }\DecValTok{1}\NormalTok{, }\DecValTok{0}\NormalTok{)}
\end{Highlighting}
\end{Shaded}

\begin{enumerate}
\def\labelenumi{\roman{enumi}.}
\setcounter{enumi}{1}
\tightlist
\item
  describe what the following variables in the data frame contain,
  \emph{trial.type}, \emph{pas}, \emph{trial}, \emph{target.contrast},
  \emph{cue}, \emph{task}, \emph{target\_type}, \emph{rt.subj},
  \emph{rt.obj}, \emph{obj.resp}, \emph{subject} and \emph{correct}.
  (That means you can ignore the rest of the variables in your
  description). For each of them, indicate and argue for what
  \texttt{class} they should be classified into, e.g.~\emph{factor},
  \emph{numeric} etc.
\end{enumerate}

\emph{trial.type} : describing whether or not the observation comes from
the introductory trials or the actual experiment. Factor.

\emph{pas} : subjective rating on how certain the participant was in
there answer on a scale from 1-4. Integer because it is ordered.

\emph{trial} : The number of trials the participant has done. Integer.

\emph{target.contrast} : contrast of the target stimuli relative to the
background. Numeric.

\emph{cue} : Which cue the participant gets at first. There are 35
different cues. Factor.

\emph{task} : how many numbers the participant is seeing single = two
numbers, pairs = four numbers, quadruplets = eight numbers. Factor.

\emph{target\_type} : Whether the participant was presented with a even
or an odd number. Factor.

\emph{rt.subj} : How fast they are to rate the \emph{pas}. Numeric.

\emph{rt.obj} : how fast they responded to the objective question
whether the number they were presented with was odd or even. Numeric.

\emph{obj.resp} : what the participant answered to whether the number
was odd or even. Factor.

\emph{subject} : participant number. Factor.

\emph{correct} : 1 if they answered correct, 0 if they answered the
question wrong. Factor.

\begin{Shaded}
\begin{Highlighting}[]
\CommentTok{\#changing all variables to the desired  }
\NormalTok{lau}\SpecialCharTok{$}\NormalTok{rt.subj }\OtherTok{\textless{}{-}} \FunctionTok{as.numeric}\NormalTok{(lau}\SpecialCharTok{$}\NormalTok{rt.subj)}
\NormalTok{lau}\SpecialCharTok{$}\NormalTok{rt.obj }\OtherTok{\textless{}{-}} \FunctionTok{as.numeric}\NormalTok{(lau}\SpecialCharTok{$}\NormalTok{rt.obj)}
\NormalTok{lau[}\FunctionTok{sapply}\NormalTok{(lau, is.character)] }\OtherTok{\textless{}{-}} \FunctionTok{lapply}\NormalTok{(lau[}\FunctionTok{sapply}\NormalTok{(lau, is.character)], as.factor)}
\NormalTok{lau}\SpecialCharTok{$}\NormalTok{cue }\OtherTok{\textless{}{-}} \FunctionTok{as.factor}\NormalTok{(lau}\SpecialCharTok{$}\NormalTok{cue)}
\NormalTok{lau}\SpecialCharTok{$}\NormalTok{subject }\OtherTok{\textless{}{-}} \FunctionTok{as.factor}\NormalTok{(lau}\SpecialCharTok{$}\NormalTok{subject)}
\NormalTok{lau}\SpecialCharTok{$}\NormalTok{correct }\OtherTok{\textless{}{-}} \FunctionTok{as.factor}\NormalTok{(lau}\SpecialCharTok{$}\NormalTok{correct)}
\NormalTok{lau}\SpecialCharTok{$}\NormalTok{target.contrast }\OtherTok{\textless{}{-}} \FunctionTok{as.numeric}\NormalTok{(lau}\SpecialCharTok{$}\NormalTok{target.contrast)}
\end{Highlighting}
\end{Shaded}

\begin{enumerate}
\def\labelenumi{\roman{enumi}.}
\setcounter{enumi}{2}
\tightlist
\item
  for the staircasing part \textbf{only}, create a plot for each subject
  where you plot the estimated function (on the \emph{target.contrast}
  range from 0-1) based on the fitted values of a model (use
  \texttt{glm}) that models \emph{correct} as dependent on
  \emph{target.contrast}. These plots will be our \_complete*-pooling\_
  model. Comment on the fits - do we have enough data to plot the
  logistic functions?
\end{enumerate}

\begin{Shaded}
\begin{Highlighting}[]
\NormalTok{staircase }\OtherTok{\textless{}{-}}\NormalTok{ lau }\SpecialCharTok{\%\textgreater{}\%} 
  \FunctionTok{filter}\NormalTok{(lau}\SpecialCharTok{$}\NormalTok{trial.type }\SpecialCharTok{==} \StringTok{\textquotesingle{}staircase\textquotesingle{}}\NormalTok{)}

\CommentTok{\# Making a function to run a model for each participant}
\NormalTok{nopoolfun }\OtherTok{\textless{}{-}} \ControlFlowTok{function}\NormalTok{(i)\{}
\NormalTok{  dat }\OtherTok{\textless{}{-}}\NormalTok{ staircase[}\FunctionTok{which}\NormalTok{(staircase}\SpecialCharTok{$}\NormalTok{subject }\SpecialCharTok{==}\NormalTok{ i),] }\CommentTok{\# subsetting the data so it only includes one participant}
\NormalTok{  model }\OtherTok{\textless{}{-}} \FunctionTok{glm}\NormalTok{(correct }\SpecialCharTok{\textasciitilde{}}\NormalTok{ target.contrast, }\AttributeTok{family =} \StringTok{\textquotesingle{}binomial\textquotesingle{}}\NormalTok{, }\AttributeTok{data =}\NormalTok{ dat) }\CommentTok{\# running a model on the data from one participant}
\NormalTok{  fitted }\OtherTok{\textless{}{-}}\NormalTok{ model}\SpecialCharTok{$}\NormalTok{fitted.values }\CommentTok{\# extracting the fitted values}
\NormalTok{  plot\_dat }\OtherTok{\textless{}{-}} \FunctionTok{data.frame}\NormalTok{(}\FunctionTok{cbind}\NormalTok{(fitted, }\StringTok{\textquotesingle{}target.contrast\textquotesingle{}} \OtherTok{=}\NormalTok{ dat}\SpecialCharTok{$}\NormalTok{target.contrast)) }\CommentTok{\# creating a data frame with the variables needed in the plot}
  
\NormalTok{  plot }\OtherTok{\textless{}{-}} \FunctionTok{ggplot}\NormalTok{(plot\_dat, }\FunctionTok{aes}\NormalTok{(}\AttributeTok{x =}\NormalTok{ target.contrast, }\AttributeTok{y =}\NormalTok{ fitted))}\SpecialCharTok{+} \CommentTok{\# plotting}
    \FunctionTok{geom\_point}\NormalTok{(}\AttributeTok{color =} \StringTok{\textquotesingle{}steelblue\textquotesingle{}}\NormalTok{) }\SpecialCharTok{+} 
    \FunctionTok{xlab}\NormalTok{(}\StringTok{\textquotesingle{}Target Contrast\textquotesingle{}}\NormalTok{) }\SpecialCharTok{+}
    \FunctionTok{ylab}\NormalTok{(}\StringTok{\textquotesingle{}Predicted\textquotesingle{}}\NormalTok{) }\SpecialCharTok{+}
    \FunctionTok{ylim}\NormalTok{(}\FunctionTok{c}\NormalTok{(}\DecValTok{0}\NormalTok{,}\DecValTok{1}\NormalTok{))}\SpecialCharTok{+}
    \FunctionTok{ggtitle}\NormalTok{(}\FunctionTok{paste0}\NormalTok{(}\StringTok{\textquotesingle{}Paticipant \textquotesingle{}}\NormalTok{, }\FunctionTok{as.character}\NormalTok{(i))) }\SpecialCharTok{+}
    \FunctionTok{theme\_minimal}\NormalTok{() }\SpecialCharTok{+}
    \FunctionTok{theme}\NormalTok{(}\AttributeTok{plot.title =} \FunctionTok{element\_text}\NormalTok{(}\AttributeTok{size =} \DecValTok{10}\NormalTok{), }\AttributeTok{axis.title=}\FunctionTok{element\_text}\NormalTok{(}\AttributeTok{size =} \DecValTok{8}\NormalTok{), }\AttributeTok{axis.text=}\FunctionTok{element\_text}\NormalTok{(}\AttributeTok{size=}\DecValTok{6}\NormalTok{))}
  
  \FunctionTok{return}\NormalTok{(plot)}
\NormalTok{\}}

\CommentTok{\# Running the function for every participant (doing it twice so the plots are nicer to look at)}
\NormalTok{subjects }\OtherTok{\textless{}{-}} \FunctionTok{c}\NormalTok{(}\DecValTok{1}\SpecialCharTok{:}\DecValTok{16}\NormalTok{)}
\NormalTok{plots }\OtherTok{\textless{}{-}} \FunctionTok{lapply}\NormalTok{(subjects, }\AttributeTok{FUN=}\NormalTok{nopoolfun)}
\end{Highlighting}
\end{Shaded}

\begin{verbatim}
## Warning: glm.fit: fitted probabilities numerically 0 or 1 occurred

## Warning: glm.fit: fitted probabilities numerically 0 or 1 occurred
\end{verbatim}

\begin{Shaded}
\begin{Highlighting}[]
\FunctionTok{do.call}\NormalTok{(grid.arrange,  plots)}
\end{Highlighting}
\end{Shaded}

\includegraphics{practical_exercise_3_files/figure-latex/unnamed-chunk-4-1.pdf}

\begin{Shaded}
\begin{Highlighting}[]
\NormalTok{subjects }\OtherTok{\textless{}{-}} \FunctionTok{c}\NormalTok{(}\DecValTok{17}\SpecialCharTok{:}\DecValTok{29}\NormalTok{)}
\NormalTok{plots }\OtherTok{\textless{}{-}} \FunctionTok{lapply}\NormalTok{(subjects, }\AttributeTok{FUN=}\NormalTok{nopoolfun)}
\end{Highlighting}
\end{Shaded}

\begin{verbatim}
## Warning: glm.fit: fitted probabilities numerically 0 or 1 occurred
\end{verbatim}

\begin{Shaded}
\begin{Highlighting}[]
\FunctionTok{do.call}\NormalTok{(grid.arrange,  plots)}
\end{Highlighting}
\end{Shaded}

\includegraphics{practical_exercise_3_files/figure-latex/unnamed-chunk-4-2.pdf}

\begin{enumerate}
\def\labelenumi{\roman{enumi}.}
\setcounter{enumi}{3}
\tightlist
\item
  on top of those plots, add the estimated functions (on the
  \emph{target.contrast} range from 0-1) for each subject based on
  partial pooling model (use \texttt{glmer} from the package
  \texttt{lme4}) where unique intercepts and slopes for
  \emph{target.contrast} are modeled for each \emph{subject}
\end{enumerate}

\begin{Shaded}
\begin{Highlighting}[]
\CommentTok{\# running a partial pooling model on the data}
\NormalTok{model }\OtherTok{\textless{}{-}} \FunctionTok{glmer}\NormalTok{(correct }\SpecialCharTok{\textasciitilde{}}\NormalTok{ target.contrast }\SpecialCharTok{+}\NormalTok{ (}\DecValTok{1} \SpecialCharTok{+}\NormalTok{ target.contrast }\SpecialCharTok{|}\NormalTok{ subject), }\AttributeTok{family =} \StringTok{\textquotesingle{}binomial\textquotesingle{}}\NormalTok{, }\AttributeTok{data =}\NormalTok{ staircase)}
\end{Highlighting}
\end{Shaded}

\begin{verbatim}
## Warning: Some predictor variables are on very different scales: consider
## rescaling
\end{verbatim}

\begin{verbatim}
## Warning in checkConv(attr(opt, "derivs"), opt$par, ctrl = control$checkConv, :
## unable to evaluate scaled gradient
\end{verbatim}

\begin{verbatim}
## Warning in checkConv(attr(opt, "derivs"), opt$par, ctrl = control$checkConv, :
## Model failed to converge: degenerate Hessian with 1 negative eigenvalues
\end{verbatim}

\begin{Shaded}
\begin{Highlighting}[]
\NormalTok{partialpoolfun }\OtherTok{\textless{}{-}} \ControlFlowTok{function}\NormalTok{(i)\{}
  \CommentTok{\# this first section is identical to the nopoolfun function, since we still want to include these data points in the plot}
\NormalTok{  dat }\OtherTok{\textless{}{-}}\NormalTok{ staircase[}\FunctionTok{which}\NormalTok{(staircase}\SpecialCharTok{$}\NormalTok{subject }\SpecialCharTok{==}\NormalTok{ i),] }
  \CommentTok{\# model for each participant}
\NormalTok{  model1 }\OtherTok{\textless{}{-}} \FunctionTok{glm}\NormalTok{(correct }\SpecialCharTok{\textasciitilde{}}\NormalTok{ target.contrast, }\AttributeTok{family =} \StringTok{\textquotesingle{}binomial\textquotesingle{}}\NormalTok{, }\AttributeTok{data =}\NormalTok{ dat)}
\NormalTok{  fitted }\OtherTok{\textless{}{-}}\NormalTok{ model1}\SpecialCharTok{$}\NormalTok{fitted.values}
\NormalTok{  plot\_dat }\OtherTok{\textless{}{-}} \FunctionTok{data.frame}\NormalTok{(}\FunctionTok{cbind}\NormalTok{(fitted, }\StringTok{\textquotesingle{}target.contrast\textquotesingle{}} \OtherTok{=}\NormalTok{ dat}\SpecialCharTok{$}\NormalTok{target.contrast))}
  
   \CommentTok{\# here a data frame is created with hypothetical target.contrast values as well as a column with the subject number}
\NormalTok{  newdf}\OtherTok{\textless{}{-}} \FunctionTok{data.frame}\NormalTok{(}\FunctionTok{cbind}\NormalTok{(}\FunctionTok{seq}\NormalTok{(}\DecValTok{0}\NormalTok{, }\DecValTok{1}\NormalTok{, }\AttributeTok{by =} \FloatTok{0.01}\NormalTok{),}\FunctionTok{rep}\NormalTok{(i)))}
  \FunctionTok{colnames}\NormalTok{(newdf) }\OtherTok{\textless{}{-}} \FunctionTok{c}\NormalTok{(}\StringTok{\textquotesingle{}target.contrast\textquotesingle{}}\NormalTok{, }\StringTok{\textquotesingle{}subject\textquotesingle{}}\NormalTok{) }\CommentTok{\#renaming the columns so it matches the variable names in the model}
  
  \CommentTok{\# predicting using the model and the data frame just created}
\NormalTok{  newdf}\SpecialCharTok{$}\NormalTok{predictmod }\OtherTok{\textless{}{-}} \FunctionTok{predict}\NormalTok{(model, }\AttributeTok{type =} \StringTok{\textquotesingle{}response\textquotesingle{}}\NormalTok{, }\AttributeTok{newdata =}\NormalTok{ newdf) }
  
\NormalTok{  plot }\OtherTok{\textless{}{-}} \FunctionTok{ggplot}\NormalTok{(plot\_dat, }\FunctionTok{aes}\NormalTok{(}\AttributeTok{x =}\NormalTok{ target.contrast, }\AttributeTok{y =}\NormalTok{ fitted)) }\SpecialCharTok{+} \CommentTok{\#plotting}
    \FunctionTok{geom\_point}\NormalTok{(}\AttributeTok{color =} \StringTok{\textquotesingle{}steelblue\textquotesingle{}}\NormalTok{) }\SpecialCharTok{+} 
    \FunctionTok{geom\_line}\NormalTok{(}\AttributeTok{data =}\NormalTok{ newdf, }\FunctionTok{aes}\NormalTok{(}\AttributeTok{x =}\NormalTok{ target.contrast, }\AttributeTok{y =}\NormalTok{ predictmod)) }\SpecialCharTok{+} 
    \FunctionTok{xlab}\NormalTok{(}\StringTok{\textquotesingle{}Target Contrast\textquotesingle{}}\NormalTok{) }\SpecialCharTok{+}
    \FunctionTok{ylab}\NormalTok{(}\StringTok{\textquotesingle{}Predicted\textquotesingle{}}\NormalTok{) }\SpecialCharTok{+}
    \FunctionTok{ylim}\NormalTok{(}\FunctionTok{c}\NormalTok{(}\DecValTok{0}\NormalTok{,}\DecValTok{1}\NormalTok{))}\SpecialCharTok{+}
    \FunctionTok{ggtitle}\NormalTok{(}\FunctionTok{paste0}\NormalTok{(}\StringTok{\textquotesingle{}Participant \textquotesingle{}}\NormalTok{, }\FunctionTok{as.character}\NormalTok{(i))) }\SpecialCharTok{+}
    \FunctionTok{theme\_minimal}\NormalTok{() }\SpecialCharTok{+}
    \FunctionTok{theme}\NormalTok{(}\AttributeTok{plot.title =} \FunctionTok{element\_text}\NormalTok{(}\AttributeTok{size =} \DecValTok{10}\NormalTok{), }\AttributeTok{axis.title=}\FunctionTok{element\_text}\NormalTok{(}\AttributeTok{size =} \DecValTok{8}\NormalTok{), }\AttributeTok{axis.text=}\FunctionTok{element\_text}\NormalTok{(}\AttributeTok{size=}\DecValTok{6}\NormalTok{))}
  
  \FunctionTok{return}\NormalTok{(plot)}
\NormalTok{\}}

\NormalTok{subjects }\OtherTok{\textless{}{-}} \FunctionTok{c}\NormalTok{(}\DecValTok{1}\SpecialCharTok{:}\DecValTok{16}\NormalTok{)}
\NormalTok{plots }\OtherTok{\textless{}{-}} \FunctionTok{lapply}\NormalTok{(subjects, }\AttributeTok{FUN=}\NormalTok{partialpoolfun)}
\FunctionTok{do.call}\NormalTok{(grid.arrange,  plots)}
\end{Highlighting}
\end{Shaded}

\includegraphics{practical_exercise_3_files/figure-latex/unnamed-chunk-6-1.pdf}

\begin{Shaded}
\begin{Highlighting}[]
\NormalTok{subjects }\OtherTok{\textless{}{-}} \FunctionTok{c}\NormalTok{(}\DecValTok{17}\SpecialCharTok{:}\DecValTok{29}\NormalTok{)}
\NormalTok{plots }\OtherTok{\textless{}{-}} \FunctionTok{lapply}\NormalTok{(subjects, }\AttributeTok{FUN=}\NormalTok{partialpoolfun)}
\FunctionTok{do.call}\NormalTok{(grid.arrange,  plots)}
\end{Highlighting}
\end{Shaded}

\includegraphics{practical_exercise_3_files/figure-latex/unnamed-chunk-6-2.pdf}

\begin{enumerate}
\def\labelenumi{\alph{enumi}.}
\setcounter{enumi}{21}
\tightlist
\item
  in your own words, describe how the partial pooling model allows for a
  better fit for each subject\\
  Partial pooling allows for a generalization of the subjects within the
  data set, since it by doing a random intercept and slope takes
  individuality into account, but it doesn't over fit the model to each
  subject like a no-pooling model would do. A complete pooling model on
  the other hand wouldn't take individuality into account and thereby
  over fit a model to each subject. By that one would in theory only be
  able to tell something about that specific subjcet and no
  generalization is then possible.
\end{enumerate}

\ldots For some reason my R doesn't want to draw lines in this
plot\ldots{}

\hypertarget{exercise-2}{%
\subsection{Exercise 2}\label{exercise-2}}

Now we \textbf{only} look at the \emph{experiment} trials
(\emph{trial.type})

\begin{enumerate}
\def\labelenumi{\arabic{enumi})}
\tightlist
\item
  Pick four subjects and plot their Quantile-Quantile (Q-Q) plots for
  the residuals of their objective response times (\emph{rt.obj}) based
  on a model where only intercept is modeled
\end{enumerate}

\begin{Shaded}
\begin{Highlighting}[]
\CommentTok{\#filtering to only get the experiment trials}
\NormalTok{lau\_exp }\OtherTok{\textless{}{-}}\NormalTok{ lau }\SpecialCharTok{\%\textgreater{}\%} 
  \FunctionTok{filter}\NormalTok{( trial.type }\SpecialCharTok{==} \StringTok{"experiment"}\NormalTok{)}

\CommentTok{\#creating a new data frame with four participants in it}
\NormalTok{df4 }\OtherTok{\textless{}{-}}\NormalTok{  lau\_exp}\SpecialCharTok{\%\textgreater{}\%} 
  \FunctionTok{filter}\NormalTok{(subject }\SpecialCharTok{==} \StringTok{\textquotesingle{}3\textquotesingle{}} \SpecialCharTok{|}\NormalTok{ subject }\SpecialCharTok{==} \StringTok{\textquotesingle{}7\textquotesingle{}} \SpecialCharTok{|}\NormalTok{ subject }\SpecialCharTok{==} \StringTok{\textquotesingle{}17\textquotesingle{}} \SpecialCharTok{|}\NormalTok{ subject }\SpecialCharTok{==} \StringTok{\textquotesingle{}28\textquotesingle{}}\NormalTok{)}


\NormalTok{qqfun }\OtherTok{\textless{}{-}} \ControlFlowTok{function}\NormalTok{(i)\{}
\NormalTok{  interceptmodel }\OtherTok{\textless{}{-}} \FunctionTok{lm}\NormalTok{(rt.obj }\SpecialCharTok{\textasciitilde{}} \DecValTok{1}\NormalTok{, }\AttributeTok{data =}\NormalTok{ df4, }\AttributeTok{subset =}\NormalTok{ subject }\SpecialCharTok{==}\NormalTok{ i)}
  \FunctionTok{qqnorm}\NormalTok{(}\FunctionTok{resid}\NormalTok{(interceptmodel), }\AttributeTok{pch =} \DecValTok{1}\NormalTok{, }\AttributeTok{frame =} \ConstantTok{FALSE}\NormalTok{)}
  \FunctionTok{qqline}\NormalTok{(}\FunctionTok{resid}\NormalTok{(interceptmodel), }\AttributeTok{col =} \StringTok{"steelblue"}\NormalTok{, }\AttributeTok{lwd =} \DecValTok{2}\NormalTok{)}
\NormalTok{\}}


\FunctionTok{qqfun}\NormalTok{(}\DecValTok{3}\NormalTok{)}
\end{Highlighting}
\end{Shaded}

\includegraphics{practical_exercise_3_files/figure-latex/unnamed-chunk-7-1.pdf}

\begin{Shaded}
\begin{Highlighting}[]
\FunctionTok{qqfun}\NormalTok{(}\DecValTok{7}\NormalTok{)}
\end{Highlighting}
\end{Shaded}

\includegraphics{practical_exercise_3_files/figure-latex/unnamed-chunk-7-2.pdf}

\begin{Shaded}
\begin{Highlighting}[]
\FunctionTok{qqfun}\NormalTok{(}\DecValTok{17}\NormalTok{)}
\end{Highlighting}
\end{Shaded}

\includegraphics{practical_exercise_3_files/figure-latex/unnamed-chunk-7-3.pdf}

\begin{Shaded}
\begin{Highlighting}[]
\FunctionTok{qqfun}\NormalTok{(}\DecValTok{28}\NormalTok{)}
\end{Highlighting}
\end{Shaded}

\includegraphics{practical_exercise_3_files/figure-latex/unnamed-chunk-7-4.pdf}
i. comment on these\\
They don't look normally distributed. They all look right skewed, some
more than others especially the last two plots.

\begin{enumerate}
\def\labelenumi{\roman{enumi}.}
\setcounter{enumi}{1}
\tightlist
\item
  does a log-transformation of the response time data improve the
  Q-Q-plots?
\end{enumerate}

\begin{Shaded}
\begin{Highlighting}[]
\CommentTok{\#Log{-}transforming the data }
\NormalTok{qqfun }\OtherTok{\textless{}{-}} \ControlFlowTok{function}\NormalTok{(i)\{}
\NormalTok{  interceptmodel }\OtherTok{\textless{}{-}} \FunctionTok{lm}\NormalTok{(}\FunctionTok{log}\NormalTok{(rt.obj) }\SpecialCharTok{\textasciitilde{}} \DecValTok{1}\NormalTok{, }\AttributeTok{data =}\NormalTok{ df4, }\AttributeTok{subset =}\NormalTok{ subject }\SpecialCharTok{==}\NormalTok{ i)}
  \FunctionTok{qqnorm}\NormalTok{(}\FunctionTok{resid}\NormalTok{(interceptmodel), }\AttributeTok{pch =} \DecValTok{1}\NormalTok{, }\AttributeTok{frame =} \ConstantTok{FALSE}\NormalTok{)}
  \FunctionTok{qqline}\NormalTok{(}\FunctionTok{resid}\NormalTok{(interceptmodel), }\AttributeTok{col =} \StringTok{"steelblue"}\NormalTok{, }\AttributeTok{lwd =} \DecValTok{2}\NormalTok{)}
\NormalTok{\}}


\FunctionTok{qqfun}\NormalTok{(}\DecValTok{3}\NormalTok{)}
\end{Highlighting}
\end{Shaded}

\includegraphics{practical_exercise_3_files/figure-latex/unnamed-chunk-8-1.pdf}

\begin{Shaded}
\begin{Highlighting}[]
\FunctionTok{qqfun}\NormalTok{(}\DecValTok{7}\NormalTok{)}
\end{Highlighting}
\end{Shaded}

\includegraphics{practical_exercise_3_files/figure-latex/unnamed-chunk-8-2.pdf}

\begin{Shaded}
\begin{Highlighting}[]
\FunctionTok{qqfun}\NormalTok{(}\DecValTok{17}\NormalTok{)}
\end{Highlighting}
\end{Shaded}

\includegraphics{practical_exercise_3_files/figure-latex/unnamed-chunk-8-3.pdf}

\begin{Shaded}
\begin{Highlighting}[]
\FunctionTok{qqfun}\NormalTok{(}\DecValTok{28}\NormalTok{)}
\end{Highlighting}
\end{Shaded}

\includegraphics{practical_exercise_3_files/figure-latex/unnamed-chunk-8-4.pdf}
Log-transforming the data made it much more normally distributed
residuals so the model fits better since normally distributed residuals
is one of the assumptions for a glm model.

\begin{enumerate}
\def\labelenumi{\arabic{enumi})}
\setcounter{enumi}{1}
\tightlist
\item
  Now do a partial pooling model modelling objective response times as
  dependent on \emph{task}? (set \texttt{REML=FALSE} in your
  \texttt{lmer}-specification)\\
\end{enumerate}

\begin{enumerate}
\def\labelenumi{\roman{enumi}.}
\tightlist
\item
  which would you include among your random effects and why? (support
  your choices with relevant measures, taking into account variance
  explained and number of parameters going into the modelling)
\end{enumerate}

\begin{Shaded}
\begin{Highlighting}[]
\NormalTok{partial1 }\OtherTok{\textless{}{-}} \FunctionTok{lmer}\NormalTok{(}\FunctionTok{log}\NormalTok{(rt.obj) }\SpecialCharTok{\textasciitilde{}}\NormalTok{ task }\SpecialCharTok{+}\NormalTok{ (}\DecValTok{1}\SpecialCharTok{|}\NormalTok{subject), }\AttributeTok{data =}\NormalTok{ lau\_exp, }\AttributeTok{REML =} \ConstantTok{FALSE}\NormalTok{)}
\NormalTok{partial2 }\OtherTok{\textless{}{-}} \FunctionTok{lmer}\NormalTok{(}\FunctionTok{log}\NormalTok{(rt.obj) }\SpecialCharTok{\textasciitilde{}}\NormalTok{ task }\SpecialCharTok{+}\NormalTok{ (}\DecValTok{1}\SpecialCharTok{|}\NormalTok{trial), }\AttributeTok{data =}\NormalTok{ lau\_exp, }\AttributeTok{REML =} \ConstantTok{FALSE}\NormalTok{)}
\NormalTok{partial3 }\OtherTok{\textless{}{-}} \FunctionTok{lmer}\NormalTok{(}\FunctionTok{log}\NormalTok{(rt.obj) }\SpecialCharTok{\textasciitilde{}}\NormalTok{ task }\SpecialCharTok{+}\NormalTok{ (}\DecValTok{1}\SpecialCharTok{|}\NormalTok{subject) }\SpecialCharTok{+}\NormalTok{ (}\DecValTok{1}\SpecialCharTok{|}\NormalTok{trial), }\AttributeTok{data =}\NormalTok{ lau\_exp, }\AttributeTok{REML =} \ConstantTok{FALSE}\NormalTok{)}
\NormalTok{partial4 }\OtherTok{\textless{}{-}} \FunctionTok{lmer}\NormalTok{(}\FunctionTok{log}\NormalTok{(rt.obj) }\SpecialCharTok{\textasciitilde{}}\NormalTok{ task }\SpecialCharTok{+}\NormalTok{ (}\DecValTok{1}\SpecialCharTok{+}\NormalTok{task}\SpecialCharTok{|}\NormalTok{subject) , }\AttributeTok{data =}\NormalTok{ lau\_exp, }\AttributeTok{REML =} \ConstantTok{FALSE}\NormalTok{)}
\NormalTok{partial5 }\OtherTok{\textless{}{-}} \FunctionTok{lmer}\NormalTok{(}\FunctionTok{log}\NormalTok{(rt.obj) }\SpecialCharTok{\textasciitilde{}}\NormalTok{ task }\SpecialCharTok{+}\NormalTok{ (}\DecValTok{1}\SpecialCharTok{+}\NormalTok{task}\SpecialCharTok{|}\NormalTok{subject) }\SpecialCharTok{+}\NormalTok{ (}\DecValTok{1}\SpecialCharTok{|}\NormalTok{trial), lau\_exp, }\AttributeTok{REML =}\NormalTok{ F)}

\FunctionTok{summary}\NormalTok{(partial1)}
\end{Highlighting}
\end{Shaded}

\begin{verbatim}
## Linear mixed model fit by maximum likelihood  ['lmerMod']
## Formula: log(rt.obj) ~ task + (1 | subject)
##    Data: lau_exp
## 
##      AIC      BIC   logLik deviance df.resid 
##  29685.3  29722.5 -14837.7  29675.3    12523 
## 
## Scaled residuals: 
##      Min       1Q   Median       3Q      Max 
## -10.8424  -0.4890  -0.0194   0.5230   7.8176 
## 
## Random effects:
##  Groups   Name        Variance Std.Dev.
##  subject  (Intercept) 0.1747   0.4180  
##  Residual             0.6186   0.7865  
## Number of obs: 12528, groups:  subject, 29
## 
## Fixed effects:
##                Estimate Std. Error t value
## (Intercept)    -0.26497    0.07857  -3.373
## taskquadruplet -0.07218    0.01721  -4.193
## tasksingles    -0.17604    0.01721 -10.227
## 
## Correlation of Fixed Effects:
##             (Intr) tskqdr
## taskqudrplt -0.110       
## tasksingles -0.110  0.500
\end{verbatim}

\begin{Shaded}
\begin{Highlighting}[]
\FunctionTok{summary}\NormalTok{(partial2)}
\end{Highlighting}
\end{Shaded}

\begin{verbatim}
## Linear mixed model fit by maximum likelihood  ['lmerMod']
## Formula: log(rt.obj) ~ task + (1 | trial)
##    Data: lau_exp
## 
##      AIC      BIC   logLik deviance df.resid 
##  32560.2  32597.3 -16275.1  32550.2    12523 
## 
## Scaled residuals: 
##      Min       1Q   Median       3Q      Max 
## -10.8657  -0.4848   0.0853   0.5704   6.7579 
## 
## Random effects:
##  Groups   Name        Variance Std.Dev.
##  trial    (Intercept) 0.02341  0.1530  
##  Residual             0.76991  0.8774  
## Number of obs: 12528, groups:  trial, 432
## 
## Fixed effects:
##                Estimate Std. Error t value
## (Intercept)    -0.26408    0.01551 -17.028
## taskquadruplet -0.07312    0.01937  -3.775
## tasksingles    -0.17776    0.01942  -9.155
## 
## Correlation of Fixed Effects:
##             (Intr) tskqdr
## taskqudrplt -0.623       
## tasksingles -0.623  0.494
\end{verbatim}

\begin{Shaded}
\begin{Highlighting}[]
\FunctionTok{summary}\NormalTok{(partial3)}
\end{Highlighting}
\end{Shaded}

\begin{verbatim}
## Linear mixed model fit by maximum likelihood  ['lmerMod']
## Formula: log(rt.obj) ~ task + (1 | subject) + (1 | trial)
##    Data: lau_exp
## 
##      AIC      BIC   logLik deviance df.resid 
##  29459.9  29504.6 -14724.0  29447.9    12522 
## 
## Scaled residuals: 
##      Min       1Q   Median       3Q      Max 
## -11.0248  -0.4909  -0.0222   0.5159   7.9938 
## 
## Random effects:
##  Groups   Name        Variance Std.Dev.
##  trial    (Intercept) 0.02977  0.1725  
##  subject  (Intercept) 0.17485  0.4182  
##  Residual             0.58883  0.7674  
## Number of obs: 12528, groups:  trial, 432; subject, 29
## 
## Fixed effects:
##                Estimate Std. Error t value
## (Intercept)    -0.26384    0.07900  -3.340
## taskquadruplet -0.07338    0.01698  -4.322
## tasksingles    -0.17824    0.01703 -10.465
## 
## Correlation of Fixed Effects:
##             (Intr) tskqdr
## taskqudrplt -0.107       
## tasksingles -0.107  0.493
\end{verbatim}

\begin{Shaded}
\begin{Highlighting}[]
\FunctionTok{summary}\NormalTok{(partial4)}
\end{Highlighting}
\end{Shaded}

\begin{verbatim}
## Linear mixed model fit by maximum likelihood  ['lmerMod']
## Formula: log(rt.obj) ~ task + (1 + task | subject)
##    Data: lau_exp
## 
##      AIC      BIC   logLik deviance df.resid 
##  29560.3  29634.7 -14770.2  29540.3    12518 
## 
## Scaled residuals: 
##      Min       1Q   Median       3Q      Max 
## -10.3756  -0.4957  -0.0211   0.5177   7.9126 
## 
## Random effects:
##  Groups   Name           Variance Std.Dev. Corr       
##  subject  (Intercept)    0.150959 0.38853             
##           taskquadruplet 0.002911 0.05395   0.41      
##           tasksingles    0.032202 0.17945   0.34 -0.72
##  Residual                0.609222 0.78053             
## Number of obs: 12528, groups:  subject, 29
## 
## Fixed effects:
##                Estimate Std. Error t value
## (Intercept)    -0.26497    0.07315  -3.622
## taskquadruplet -0.07218    0.01980  -3.645
## tasksingles    -0.17604    0.03745  -4.701
## 
## Correlation of Fixed Effects:
##             (Intr) tskqdr
## taskqudrplt  0.102       
## tasksingles  0.244 -0.128
\end{verbatim}

\begin{Shaded}
\begin{Highlighting}[]
\FunctionTok{summary}\NormalTok{(partial5)}
\end{Highlighting}
\end{Shaded}

\begin{verbatim}
## Linear mixed model fit by maximum likelihood  ['lmerMod']
## Formula: log(rt.obj) ~ task + (1 + task | subject) + (1 | trial)
##    Data: lau_exp
## 
##      AIC      BIC   logLik deviance df.resid 
##  29327.7  29409.5 -14652.8  29305.7    12517 
## 
## Scaled residuals: 
##      Min       1Q   Median       3Q      Max 
## -10.5363  -0.4958  -0.0279   0.5119   8.0799 
## 
## Random effects:
##  Groups   Name           Variance Std.Dev. Corr       
##  trial    (Intercept)    0.02995  0.17307             
##  subject  (Intercept)    0.14918  0.38623             
##           taskquadruplet 0.00400  0.06325   0.35      
##           tasksingles    0.03227  0.17964   0.38 -0.58
##  Residual                0.57917  0.76103             
## Number of obs: 12528, groups:  trial, 432; subject, 29
## 
## Fixed effects:
##                Estimate Std. Error t value
## (Intercept)    -0.26407    0.07317  -3.609
## taskquadruplet -0.07225    0.02054  -3.518
## tasksingles    -0.17868    0.03739  -4.778
## 
## Correlation of Fixed Effects:
##             (Intr) tskqdr
## taskqudrplt  0.099       
## tasksingles  0.279 -0.114
\end{verbatim}

\begin{Shaded}
\begin{Highlighting}[]
\NormalTok{model\_text }\OtherTok{\textless{}{-}} \FunctionTok{c}\NormalTok{(}\StringTok{"subject"}\NormalTok{, }\StringTok{"trial"}\NormalTok{, }\StringTok{"subject og trial"}\NormalTok{, }\StringTok{"task{-}subject"}\NormalTok{,}\StringTok{"task{-}subject og trial"}\NormalTok{)}
\CommentTok{\#sigma explains variance}
\NormalTok{sigmas }\OtherTok{\textless{}{-}} \FunctionTok{c}\NormalTok{(}\FunctionTok{sigma}\NormalTok{(partial1),}\FunctionTok{sigma}\NormalTok{(partial2),}\FunctionTok{sigma}\NormalTok{(partial3), }\FunctionTok{sigma}\NormalTok{(partial4), }\FunctionTok{sigma}\NormalTok{(partial5))}
\NormalTok{AIC }\OtherTok{\textless{}{-}} \FunctionTok{c}\NormalTok{(}\FunctionTok{AIC}\NormalTok{(partial1), }\FunctionTok{AIC}\NormalTok{(partial2), }\FunctionTok{AIC}\NormalTok{(partial3), }\FunctionTok{AIC}\NormalTok{(partial4), }\FunctionTok{AIC}\NormalTok{(partial5))}
\NormalTok{mtable }\OtherTok{\textless{}{-}} \FunctionTok{as\_tibble}\NormalTok{(}\FunctionTok{cbind}\NormalTok{(model\_text,sigmas,AIC))}
\NormalTok{mtable}
\end{Highlighting}
\end{Shaded}

\begin{verbatim}
## # A tibble: 5 x 3
##   model_text            sigmas            AIC             
##   <chr>                 <chr>             <chr>           
## 1 subject               0.786510836267907 29685.3126188652
## 2 trial                 0.877443929266087 32560.1503149819
## 3 subject og trial      0.767353686772583 29459.9382110624
## 4 task-subject          0.780526761806287 29560.3359953566
## 5 task-subject og trial 0.761030007720908 29327.6815811547
\end{verbatim}

If the partial5 model was too complex the AIC would `punish' the model,
but it doesn't. The model where \emph{log(rt.obj) \textasciitilde{} task
+ (1+task\textbar subject) + (1\textbar trial)} is the one with both the
lowest AIC and the lowest sigma, which means that it's the one with most
variance explained.

\begin{enumerate}
\def\labelenumi{\roman{enumi}.}
\setcounter{enumi}{1}
\tightlist
\item
  explain in your own words what your chosen models says about response
  times between the different tasks\\
  Both in the tasks where it's quadruplets and singles the reaction time
  for the participant is shorter compared to the paired task. This we
  see on the output from the partial5 model since both estimates are
  negative (these numbers are on the log scale). By adding subject and
  trial as random intercepts we make sure that they have their own
  baseline reaction time.
\end{enumerate}

\begin{enumerate}
\def\labelenumi{\arabic{enumi})}
\setcounter{enumi}{2}
\tightlist
\item
  Now add \emph{pas} and its interaction with \emph{task} to the fixed
  effects
\end{enumerate}

\begin{Shaded}
\begin{Highlighting}[]
\NormalTok{interactio }\OtherTok{\textless{}{-}} \FunctionTok{lm}\NormalTok{(}\FunctionTok{log}\NormalTok{(rt.obj) }\SpecialCharTok{\textasciitilde{}}\NormalTok{ task}\SpecialCharTok{*}\NormalTok{pas, lau\_exp)}
\end{Highlighting}
\end{Shaded}

\begin{enumerate}
\def\labelenumi{\roman{enumi}.}
\tightlist
\item
  how many types of group intercepts (random effects) can you add
  without ending up with convergence issues or singular fits?
\end{enumerate}

\begin{Shaded}
\begin{Highlighting}[]
\NormalTok{interactio\_mix }\OtherTok{\textless{}{-}} \FunctionTok{lmer}\NormalTok{(}\FunctionTok{log}\NormalTok{(rt.obj) }\SpecialCharTok{\textasciitilde{}}\NormalTok{ task}\SpecialCharTok{*}\NormalTok{pas }\SpecialCharTok{+}\NormalTok{ (}\DecValTok{1}\SpecialCharTok{|}\NormalTok{subject), }\AttributeTok{data =}\NormalTok{ lau\_exp, }\AttributeTok{REML =}\NormalTok{ F)}
\NormalTok{interactio\_mix1 }\OtherTok{\textless{}{-}} \FunctionTok{lmer}\NormalTok{(}\FunctionTok{log}\NormalTok{(rt.obj) }\SpecialCharTok{\textasciitilde{}}\NormalTok{ task}\SpecialCharTok{*}\NormalTok{pas }\SpecialCharTok{+}\NormalTok{ (}\DecValTok{1}\SpecialCharTok{|}\NormalTok{trial) }\SpecialCharTok{+}\NormalTok{ (}\DecValTok{1}\SpecialCharTok{|}\NormalTok{subject), }\AttributeTok{data =}\NormalTok{ lau\_exp, }\AttributeTok{REML =}\NormalTok{ F)}
\NormalTok{interactio\_mix2 }\OtherTok{\textless{}{-}} \FunctionTok{lmer}\NormalTok{(}\FunctionTok{log}\NormalTok{(rt.obj) }\SpecialCharTok{\textasciitilde{}}\NormalTok{ task}\SpecialCharTok{*}\NormalTok{pas }\SpecialCharTok{+}\NormalTok{ (}\DecValTok{1}\SpecialCharTok{|}\NormalTok{trial) }\SpecialCharTok{+}\NormalTok{ (}\DecValTok{1}\SpecialCharTok{|}\NormalTok{subject) }\SpecialCharTok{+}\NormalTok{ (}\DecValTok{1}\SpecialCharTok{|}\NormalTok{cue), }\AttributeTok{data =}\NormalTok{ lau\_exp, }\AttributeTok{REML =}\NormalTok{ F)}
\end{Highlighting}
\end{Shaded}

\begin{enumerate}
\def\labelenumi{\roman{enumi}.}
\setcounter{enumi}{1}
\tightlist
\item
  create a model by adding random intercepts (without modelling slopes)
  that results in a singular fit - then use
  \texttt{print(VarCorr(\textless{}your.model\textgreater{}),\ comp=\textquotesingle{}Variance\textquotesingle{})}
  to inspect the variance vector - explain why the fit is singular
  (Hint: read the first paragraph under details in the help for
  \texttt{isSingular})
\end{enumerate}

\begin{Shaded}
\begin{Highlighting}[]
\NormalTok{sing\_fit }\OtherTok{\textless{}{-}} \FunctionTok{lmer}\NormalTok{(}\FunctionTok{log}\NormalTok{(rt.obj) }\SpecialCharTok{\textasciitilde{}}\NormalTok{ task}\SpecialCharTok{*}\NormalTok{pas }\SpecialCharTok{+}\NormalTok{ (}\DecValTok{1}\SpecialCharTok{|}\NormalTok{trial) }\SpecialCharTok{+}\NormalTok{ (}\DecValTok{1}\SpecialCharTok{|}\NormalTok{subject) }\SpecialCharTok{+}\NormalTok{ (}\DecValTok{1}\SpecialCharTok{|}\NormalTok{cue) }\SpecialCharTok{+}\NormalTok{ (}\DecValTok{1}\SpecialCharTok{|}\NormalTok{target.contrast), }\AttributeTok{data =}\NormalTok{ lau\_exp, }\AttributeTok{REML =}\NormalTok{ F)}
\end{Highlighting}
\end{Shaded}

\begin{verbatim}
## boundary (singular) fit: see ?isSingular
\end{verbatim}

\begin{Shaded}
\begin{Highlighting}[]
\FunctionTok{print}\NormalTok{(}\FunctionTok{VarCorr}\NormalTok{(sing\_fit), }\AttributeTok{comp=}\StringTok{\textquotesingle{}Variance\textquotesingle{}}\NormalTok{)}
\end{Highlighting}
\end{Shaded}

\begin{verbatim}
##  Groups          Name        Variance  
##  trial           (Intercept) 2.6842e-02
##  cue             (Intercept) 4.0356e-03
##  target.contrast (Intercept) 4.8074e-11
##  subject         (Intercept) 1.6287e-01
##  Residual                    5.7156e-01
\end{verbatim}

\begin{enumerate}
\def\labelenumi{\roman{enumi}.}
\setcounter{enumi}{2}
\tightlist
\item
  in your own words - how could you explain why your model would result
  in a singular fit?\\
  The fit is singular because the random-effects variances are close to
  0. This is because we have too many random effects. They do each not
  explain a lot of variance in the data, this can also occour when the
  variables are higly correlated.
\end{enumerate}

\hypertarget{exercise-3}{%
\subsection{Exercise 3}\label{exercise-3}}

\begin{enumerate}
\def\labelenumi{\arabic{enumi})}
\tightlist
\item
  Initialize a new data frame, \texttt{data.count}. \emph{count} should
  indicate the number of times they categorized their experience as
  \emph{pas} 1-4 for each \emph{task}. I.e. the data frame would have
  for subject 1: for task:singles, pas1 was used \# times, pas2 was used
  \# times, pas3 was used \# times and pas4 was used \# times. You would
  then do the same for task:pairs and task:quadruplet
\end{enumerate}

\begin{Shaded}
\begin{Highlighting}[]
\NormalTok{data.count }\OtherTok{\textless{}{-}}\NormalTok{ lau\_exp }\SpecialCharTok{\%\textgreater{}\%} 
  \FunctionTok{group\_by}\NormalTok{(subject, pas, task) }\SpecialCharTok{\%\textgreater{}\%} 
  \FunctionTok{summarise}\NormalTok{(}\AttributeTok{count=}\FunctionTok{n}\NormalTok{())}
\end{Highlighting}
\end{Shaded}

\begin{verbatim}
## `summarise()` has grouped output by 'subject', 'pas'. You can override using the `.groups` argument.
\end{verbatim}

\begin{Shaded}
\begin{Highlighting}[]
\NormalTok{data.count}\SpecialCharTok{$}\NormalTok{pas }\OtherTok{\textless{}{-}} \FunctionTok{as.factor}\NormalTok{(data.count}\SpecialCharTok{$}\NormalTok{pas)}
\end{Highlighting}
\end{Shaded}

\begin{enumerate}
\def\labelenumi{\arabic{enumi})}
\setcounter{enumi}{1}
\tightlist
\item
  Now fit a multilevel model that models a unique ``slope'' for
  \emph{pas} for each \emph{subject} with the interaction between
  \emph{pas} and \emph{task} and their main effects being modeled

  \begin{enumerate}
  \def\labelenumii{\roman{enumii}.}
  \tightlist
  \item
    which family should be used?
  \end{enumerate}
\end{enumerate}

The `poisson' should be used as we are modeling count data

\begin{Shaded}
\begin{Highlighting}[]
\NormalTok{poisson\_model }\OtherTok{\textless{}{-}} \FunctionTok{glmer}\NormalTok{(count }\SpecialCharTok{\textasciitilde{}}\NormalTok{ pas}\SpecialCharTok{*}\NormalTok{task }\SpecialCharTok{+}\NormalTok{ (}\DecValTok{1} \SpecialCharTok{+}\NormalTok{ pas}\SpecialCharTok{|}\NormalTok{subject), }\AttributeTok{family =} \StringTok{"poisson"}\NormalTok{, }\AttributeTok{data =}\NormalTok{ data.count, }\FunctionTok{glmerControl}\NormalTok{(}\AttributeTok{optimizer=}\StringTok{"bobyqa"}\NormalTok{))}
\FunctionTok{summary}\NormalTok{(poisson\_model)}
\end{Highlighting}
\end{Shaded}

\begin{verbatim}
## Generalized linear mixed model fit by maximum likelihood (Laplace
##   Approximation) [glmerMod]
##  Family: poisson  ( log )
## Formula: count ~ pas * task + (1 + pas | subject)
##    Data: data.count
## Control: glmerControl(optimizer = "bobyqa")
## 
##      AIC      BIC   logLik deviance df.resid 
##   2779.0   2861.8  -1367.5   2735.0      297 
## 
## Scaled residuals: 
##     Min      1Q  Median      3Q     Max 
## -4.2797 -0.8249 -0.0760  0.7367  6.4498 
## 
## Random effects:
##  Groups  Name        Variance Std.Dev. Corr             
##  subject (Intercept) 0.5628   0.7502                    
##          pas2        0.6562   0.8101   -0.73            
##          pas3        1.1803   1.0864   -0.91  0.80      
##          pas4        3.8601   1.9647   -0.91  0.50  0.80
## Number of obs: 319, groups:  subject, 29
## 
## Fixed effects:
##                     Estimate Std. Error z value Pr(>|z|)    
## (Intercept)          3.61415    0.14254  25.355  < 2e-16 ***
## pas2                -0.04098    0.15647  -0.262 0.793418    
## pas3                -0.37077    0.20907  -1.773 0.076164 .  
## pas4                -0.94171    0.37739  -2.495 0.012585 *  
## taskquadruplet       0.06071    0.03732   1.627 0.103762    
## tasksingles         -0.23490    0.04031  -5.828 5.62e-09 ***
## pas2:taskquadruplet -0.04411    0.05520  -0.799 0.424197    
## pas3:taskquadruplet -0.11069    0.06365  -1.739 0.082015 .  
## pas4:taskquadruplet -0.11778    0.06279  -1.876 0.060665 .  
## pas2:tasksingles     0.17454    0.05783   3.018 0.002543 ** 
## pas3:tasksingles     0.23754    0.06457   3.679 0.000234 ***
## pas4:tasksingles     0.59798    0.06098   9.807  < 2e-16 ***
## ---
## Signif. codes:  0 '***' 0.001 '**' 0.01 '*' 0.05 '.' 0.1 ' ' 1
## 
## Correlation of Fixed Effects:
##             (Intr) pas2   pas3   pas4   tskqdr tsksng ps2:tskq ps3:tskq
## pas2        -0.726                                                     
## pas3        -0.893  0.772                                              
## pas4        -0.880  0.477  0.764                                       
## taskqudrplt -0.135  0.123  0.092  0.051                                
## tasksingles -0.125  0.114  0.085  0.047  0.477                         
## ps2:tskqdrp  0.091 -0.180 -0.062 -0.034 -0.676 -0.322                  
## ps3:tskqdrp  0.079 -0.072 -0.149 -0.029 -0.586 -0.280  0.396           
## ps4:tskqdrp  0.080 -0.073 -0.055 -0.079 -0.594 -0.283  0.402    0.348  
## ps2:tsksngl  0.087 -0.171 -0.059 -0.033 -0.332 -0.697  0.486    0.195  
## ps3:tsksngl  0.078 -0.071 -0.151 -0.030 -0.298 -0.624  0.201    0.485  
## ps4:tsksngl  0.082 -0.075 -0.056 -0.085 -0.315 -0.661  0.213    0.185  
##             ps4:tskq ps2:tsks ps3:tsks
## pas2                                  
## pas3                                  
## pas4                                  
## taskqudrplt                           
## tasksingles                           
## ps2:tskqdrp                           
## ps3:tskqdrp                           
## ps4:tskqdrp                           
## ps2:tsksngl  0.198                    
## ps3:tsksngl  0.177    0.435           
## ps4:tsksngl  0.509    0.461    0.413
\end{verbatim}

\begin{enumerate}
\def\labelenumi{\roman{enumi}.}
\setcounter{enumi}{1}
\item
  why is a slope for \emph{pas} not really being modeled?\\
  If you look at pas as a fixed effect on its own, it does not make much
  sense; it does not make sense to use the ratings of pas to predict
  count. We are interested in the interaction between pas and task,
  since we can only use pas to say something in relation to task.
\item
  if you get a convergence error, try another algorithm (the default is
  the \emph{Nelder\_Mead}) - try (\emph{bobyqa}) for which the
  \texttt{dfoptim} package is needed. In \texttt{glmer}, you can add the
  following for the \texttt{control} argument:
  \texttt{glmerControl(optimizer="bobyqa")} (if you are interested, also
  have a look at the function \texttt{allFit})
\item
  when you have a converging fit - fit a model with only the main
  effects of \emph{pas} and \emph{task}. Compare this with the model
  that also includes the interaction
\end{enumerate}

\begin{Shaded}
\begin{Highlighting}[]
\NormalTok{poisson\_model2 }\OtherTok{\textless{}{-}} \FunctionTok{glmer}\NormalTok{(count }\SpecialCharTok{\textasciitilde{}}\NormalTok{ pas }\SpecialCharTok{+}\NormalTok{ task }\SpecialCharTok{+}\NormalTok{ (}\DecValTok{1} \SpecialCharTok{+}\NormalTok{ pas}\SpecialCharTok{|}\NormalTok{subject), }\AttributeTok{family =} \StringTok{"poisson"}\NormalTok{, }\AttributeTok{data =}\NormalTok{ data.count, }\FunctionTok{glmerControl}\NormalTok{(}\AttributeTok{optimizer=}\StringTok{"bobyqa"}\NormalTok{))}

\NormalTok{modelos }\OtherTok{\textless{}{-}} \FunctionTok{c}\NormalTok{(}\StringTok{"poisson\_model"}\NormalTok{, }\StringTok{"poisson\_model2"}\NormalTok{)}

\CommentTok{\#finding residual standard deviation}
\NormalTok{residual\_standard\_deviations }\OtherTok{\textless{}{-}} \FunctionTok{c}\NormalTok{(}\FunctionTok{sigma}\NormalTok{(poisson\_model), }\FunctionTok{sigma}\NormalTok{(poisson\_model2))}

\CommentTok{\#finding AIC}
\NormalTok{AIC\_values }\OtherTok{\textless{}{-}} \FunctionTok{c}\NormalTok{(}\FunctionTok{AIC}\NormalTok{(poisson\_model), }\FunctionTok{AIC}\NormalTok{(poisson\_model2))}

\CommentTok{\#combining it all in a tibble}
\FunctionTok{as.tibble}\NormalTok{(}\FunctionTok{cbind}\NormalTok{(modelos, residual\_standard\_deviations, AIC\_values))}
\end{Highlighting}
\end{Shaded}

\begin{verbatim}
## Warning: `as.tibble()` was deprecated in tibble 2.0.0.
## Please use `as_tibble()` instead.
## The signature and semantics have changed, see `?as_tibble`.
## This warning is displayed once every 8 hours.
## Call `lifecycle::last_warnings()` to see where this warning was generated.
\end{verbatim}

\begin{verbatim}
## # A tibble: 2 x 3
##   modelos        residual_standard_deviations AIC_values      
##   <chr>          <chr>                        <chr>           
## 1 poisson_model  1                            2778.98742340758
## 2 poisson_model2 1                            2933.26390057674
\end{verbatim}

\begin{enumerate}
\def\labelenumi{\alph{enumi}.}
\setcounter{enumi}{21}
\tightlist
\item
  indicate which of the two models, you would choose and why\\
  The interaction model has the lowest AIC, 2778. Besides that both
  models has a residual standard deviation of 1, which is because you
  cannot make a standard deviation on a logistic model.
\end{enumerate}

It does not make sense to include pas without it having an interaction
with count since pas alone doesn't mean anything.

\begin{enumerate}
\def\labelenumi{\roman{enumi}.}
\setcounter{enumi}{5}
\tightlist
\item
  based on your chosen model - write a short report on what this says
  about the distribution of ratings as dependent on \emph{pas} and
  \emph{task}
\end{enumerate}

\begin{Shaded}
\begin{Highlighting}[]
\NormalTok{model\_text }\OtherTok{\textless{}{-}} \FunctionTok{c}\NormalTok{(}\StringTok{"Intercept"}\NormalTok{, }\StringTok{"Quadruplet"}\NormalTok{, }\StringTok{"Singles"}\NormalTok{)}
\NormalTok{estimates }\OtherTok{\textless{}{-}} \FunctionTok{c}\NormalTok{(poisson\_model}\SpecialCharTok{@}\NormalTok{beta[}\DecValTok{1}\NormalTok{], poisson\_model}\SpecialCharTok{@}\NormalTok{beta[}\DecValTok{5}\NormalTok{], poisson\_model}\SpecialCharTok{@}\NormalTok{beta[}\DecValTok{6}\NormalTok{])}
\NormalTok{estimatesExp }\OtherTok{\textless{}{-}} \FunctionTok{c}\NormalTok{(}\FunctionTok{exp}\NormalTok{(poisson\_model}\SpecialCharTok{@}\NormalTok{beta[}\DecValTok{1}\NormalTok{]), }\FunctionTok{exp}\NormalTok{(poisson\_model}\SpecialCharTok{@}\NormalTok{beta[}\DecValTok{5}\NormalTok{]), }\FunctionTok{exp}\NormalTok{(poisson\_model}\SpecialCharTok{@}\NormalTok{beta[}\DecValTok{6}\NormalTok{]))}
\NormalTok{tasktable }\OtherTok{\textless{}{-}} \FunctionTok{as\_tibble}\NormalTok{(}\FunctionTok{cbind}\NormalTok{(model\_text,estimates,estimatesExp))}
\NormalTok{tasktable}
\end{Highlighting}
\end{Shaded}

\begin{verbatim}
## # A tibble: 3 x 3
##   model_text estimates          estimatesExp    
##   <chr>      <chr>              <chr>           
## 1 Intercept  3.61414675437776   37.1196602221413
## 2 Quadruplet 0.0607086477095806 1.06258928126238
## 3 Singles    -0.234903744627158 0.7906469499831
\end{verbatim}

\begin{Shaded}
\begin{Highlighting}[]
\NormalTok{model\_text2 }\OtherTok{\textless{}{-}} \FunctionTok{c}\NormalTok{(}\StringTok{"Pas 2, quadruplet"}\NormalTok{, }\StringTok{"Pas 3, quadruplet"}\NormalTok{, }\StringTok{"Pas 4, quadruplet"}\NormalTok{,}\StringTok{"Pas 2, singles"}\NormalTok{, }\StringTok{"Pas 3, singles"}\NormalTok{, }\StringTok{"Pas 4, singles"}\NormalTok{)}
\NormalTok{estimates2 }\OtherTok{\textless{}{-}} \FunctionTok{c}\NormalTok{(poisson\_model}\SpecialCharTok{@}\NormalTok{beta[}\DecValTok{7}\NormalTok{], poisson\_model}\SpecialCharTok{@}\NormalTok{beta[}\DecValTok{8}\NormalTok{], poisson\_model}\SpecialCharTok{@}\NormalTok{beta[}\DecValTok{9}\NormalTok{], poisson\_model}\SpecialCharTok{@}\NormalTok{beta[}\DecValTok{10}\NormalTok{], poisson\_model}\SpecialCharTok{@}\NormalTok{beta[}\DecValTok{11}\NormalTok{], poisson\_model}\SpecialCharTok{@}\NormalTok{beta[}\DecValTok{12}\NormalTok{])}
\NormalTok{estimatesExp2 }\OtherTok{\textless{}{-}} \FunctionTok{c}\NormalTok{(}\FunctionTok{exp}\NormalTok{(poisson\_model}\SpecialCharTok{@}\NormalTok{beta[}\DecValTok{7}\NormalTok{]), }\FunctionTok{exp}\NormalTok{(poisson\_model}\SpecialCharTok{@}\NormalTok{beta[}\DecValTok{8}\NormalTok{]), }\FunctionTok{exp}\NormalTok{(poisson\_model}\SpecialCharTok{@}\NormalTok{beta[}\DecValTok{9}\NormalTok{]), }\FunctionTok{exp}\NormalTok{(poisson\_model}\SpecialCharTok{@}\NormalTok{beta[}\DecValTok{10}\NormalTok{]), }\FunctionTok{exp}\NormalTok{(poisson\_model}\SpecialCharTok{@}\NormalTok{beta[}\DecValTok{11}\NormalTok{]), }\FunctionTok{exp}\NormalTok{(poisson\_model}\SpecialCharTok{@}\NormalTok{beta[}\DecValTok{12}\NormalTok{]))}
\NormalTok{interactiontable }\OtherTok{\textless{}{-}} \FunctionTok{as\_tibble}\NormalTok{(}\FunctionTok{cbind}\NormalTok{(model\_text2,estimates2,estimatesExp2))}
\NormalTok{interactiontable}
\end{Highlighting}
\end{Shaded}

\begin{verbatim}
## # A tibble: 6 x 3
##   model_text2       estimates2         estimatesExp2    
##   <chr>             <chr>              <chr>            
## 1 Pas 2, quadruplet -0.044110412632432 0.95684830350036 
## 2 Pas 3, quadruplet -0.110691997510834 0.895214434745132
## 3 Pas 4, quadruplet -0.117784926708673 0.88888720795511 
## 4 Pas 2, singles    0.174541036125271  1.19069960308055 
## 5 Pas 3, singles    0.237540984928311  1.2681269698186  
## 6 Pas 4, singles    0.597975351136592  1.8184333817306
\end{verbatim}

First, let's look at the task estimates. Our intercept signifies pas1,
and pas1 is the most uncertain rating. When keeping the rest of the
variables constant and changing task from pairs to quadruplet it
increases the count (positive estimate, 0.06). On the other hand looking
at the task to singles it decreases in count (negative estimate -0.23).
This means that the harder the task the more the count of pas1 increases
while easier tasks decreases the count of pas1.Seen on the output from
the \emph{poisson\_model} the easier the task (i.e.~the fewer numbers in
a task) the more sure the subject is on whether or not they've answered
the question correctly, ergo the higher pas. And vice versa the
estimates for the quadruplets task are all negative, which means the
subjects in these cases have been more unsure of their answers.

Looking at the exponential estimates in the latter table we see that
count of pas1 increases by 6\% compared to the baseline when going from
the pairs task to the quadruplet task. Subjects are less sure in the
quadruplet task, which means a higher count of pas1 compared to the
singles task where the count of pas4 for the subjects is much higher.
This makes much sense, since the pairs tasks are easier than the
quadruplet tasks, since it has fewer numbers in it.

\begin{enumerate}
\def\labelenumi{\roman{enumi}.}
\setcounter{enumi}{6}
\tightlist
\item
  include a plot that shows the estimated amount of ratings for four
  subjects of your choosing
\end{enumerate}

\begin{Shaded}
\begin{Highlighting}[]
\NormalTok{dfplot }\OtherTok{\textless{}{-}}\NormalTok{  data.count}\SpecialCharTok{\%\textgreater{}\%} 
  \FunctionTok{filter}\NormalTok{(subject }\SpecialCharTok{==} \StringTok{\textquotesingle{}3\textquotesingle{}} \SpecialCharTok{|}\NormalTok{ subject }\SpecialCharTok{==} \StringTok{\textquotesingle{}7\textquotesingle{}} \SpecialCharTok{|}\NormalTok{ subject }\SpecialCharTok{==} \StringTok{\textquotesingle{}17\textquotesingle{}} \SpecialCharTok{|}\NormalTok{ subject }\SpecialCharTok{==} \StringTok{\textquotesingle{}28\textquotesingle{}}\NormalTok{)}

\NormalTok{dfplot}\SpecialCharTok{$}\NormalTok{predicted }\OtherTok{\textless{}{-}} \FunctionTok{predict}\NormalTok{(poisson\_model, }\AttributeTok{newdata=}\NormalTok{dfplot)}

\FunctionTok{ggplot}\NormalTok{(dfplot, }\FunctionTok{aes}\NormalTok{(}\AttributeTok{x =}\NormalTok{ pas, }\AttributeTok{y =} \FunctionTok{exp}\NormalTok{(predicted), }\AttributeTok{fill=}\NormalTok{pas)) }\SpecialCharTok{+} 
  \FunctionTok{geom\_bar}\NormalTok{(}\AttributeTok{stat =} \StringTok{\textquotesingle{}identity\textquotesingle{}}\NormalTok{) }\SpecialCharTok{+} 
  \FunctionTok{facet\_wrap}\NormalTok{(}\SpecialCharTok{\textasciitilde{}}\NormalTok{ subject)}\SpecialCharTok{+}
  \FunctionTok{theme\_minimal}\NormalTok{()}
\end{Highlighting}
\end{Shaded}

\includegraphics{practical_exercise_3_files/figure-latex/unnamed-chunk-18-1.pdf}

\begin{enumerate}
\def\labelenumi{\arabic{enumi})}
\setcounter{enumi}{2}
\tightlist
\item
  Finally, fit a multilevel model that models \emph{correct} as
  dependent on \emph{task} with a unique intercept for each
  \emph{subject}
\end{enumerate}

\begin{Shaded}
\begin{Highlighting}[]
\NormalTok{last\_one }\OtherTok{\textless{}{-}} \FunctionTok{glmer}\NormalTok{(correct}\SpecialCharTok{\textasciitilde{}}\NormalTok{task }\SpecialCharTok{+}\NormalTok{ (}\DecValTok{1}\SpecialCharTok{|}\NormalTok{subject), }\AttributeTok{data =}\NormalTok{lau\_exp, }\AttributeTok{family =} \StringTok{\textquotesingle{}binomial\textquotesingle{}}\NormalTok{)}
\FunctionTok{summary}\NormalTok{(last\_one)}
\end{Highlighting}
\end{Shaded}

\begin{verbatim}
## Generalized linear mixed model fit by maximum likelihood (Laplace
##   Approximation) [glmerMod]
##  Family: binomial  ( logit )
## Formula: correct ~ task + (1 | subject)
##    Data: lau_exp
## 
##      AIC      BIC   logLik deviance df.resid 
##  13636.3  13666.0  -6814.1  13628.3    12524 
## 
## Scaled residuals: 
##     Min      1Q  Median      3Q     Max 
## -3.7450 -1.0599  0.4791  0.6483  0.9795 
## 
## Random effects:
##  Groups  Name        Variance Std.Dev.
##  subject (Intercept) 0.3622   0.6018  
## Number of obs: 12528, groups:  subject, 29
## 
## Fixed effects:
##                Estimate Std. Error z value Pr(>|z|)    
## (Intercept)     1.11896    0.11773   9.504  < 2e-16 ***
## taskquadruplet -0.07496    0.05082  -1.475  0.14019    
## tasksingles     0.16603    0.05218   3.182  0.00146 ** 
## ---
## Signif. codes:  0 '***' 0.001 '**' 0.01 '*' 0.05 '.' 0.1 ' ' 1
## 
## Correlation of Fixed Effects:
##             (Intr) tskqdr
## taskqudrplt -0.220       
## tasksingles -0.213  0.494
\end{verbatim}

\begin{enumerate}
\def\labelenumi{\roman{enumi}.}
\item
  does \emph{task} explain performance?\\
  It seems as if task has some explanatory power since task singles is
  significantly different from task pairs.
\item
  add \emph{pas} as a main effect on top of \emph{task} - what are the
  consequences of that?
\end{enumerate}

\begin{Shaded}
\begin{Highlighting}[]
\NormalTok{last\_one2 }\OtherTok{\textless{}{-}} \FunctionTok{glmer}\NormalTok{(correct}\SpecialCharTok{\textasciitilde{}}\NormalTok{task }\SpecialCharTok{+}\NormalTok{ pas }\SpecialCharTok{+}\NormalTok{ (}\DecValTok{1}\SpecialCharTok{|}\NormalTok{subject), }\AttributeTok{data =}\NormalTok{lau\_exp, }\AttributeTok{family =} \StringTok{\textquotesingle{}binomial\textquotesingle{}}\NormalTok{)}
\FunctionTok{summary}\NormalTok{(last\_one2)}
\end{Highlighting}
\end{Shaded}

\begin{verbatim}
## Generalized linear mixed model fit by maximum likelihood (Laplace
##   Approximation) [glmerMod]
##  Family: binomial  ( logit )
## Formula: correct ~ task + pas + (1 | subject)
##    Data: lau_exp
## 
##      AIC      BIC   logLik deviance df.resid 
##  12258.2  12295.4  -6124.1  12248.2    12523 
## 
## Scaled residuals: 
##     Min      1Q  Median      3Q     Max 
## -5.7622 -0.7280  0.3227  0.5954  1.6353 
## 
## Random effects:
##  Groups  Name        Variance Std.Dev.
##  subject (Intercept) 0.2424   0.4923  
## Number of obs: 12528, groups:  subject, 29
## 
## Fixed effects:
##                Estimate Std. Error z value Pr(>|z|)    
## (Intercept)    -0.81547    0.11200  -7.281 3.31e-13 ***
## taskquadruplet -0.02632    0.05372  -0.490    0.624    
## tasksingles    -0.03492    0.05569  -0.627    0.531    
## pas             0.94641    0.02788  33.951  < 2e-16 ***
## ---
## Signif. codes:  0 '***' 0.001 '**' 0.01 '*' 0.05 '.' 0.1 ' ' 1
## 
## Correlation of Fixed Effects:
##             (Intr) tskqdr tsksng
## taskqudrplt -0.253              
## tasksingles -0.190  0.488       
## pas         -0.461  0.020 -0.095
\end{verbatim}

Task is no longer significant when adding pas as a main effect. It seems
like \emph{pas} explains whether or not the subject answered the task
correctly better than task. ???

\begin{enumerate}
\def\labelenumi{\roman{enumi}.}
\setcounter{enumi}{2}
\tightlist
\item
  now fit a multilevel model that models \emph{correct} as dependent on
  \emph{pas} with a unique intercept for each \emph{subject}
\end{enumerate}

\begin{Shaded}
\begin{Highlighting}[]
\NormalTok{last\_one3 }\OtherTok{\textless{}{-}} \FunctionTok{glmer}\NormalTok{(correct}\SpecialCharTok{\textasciitilde{}}\NormalTok{pas }\SpecialCharTok{+}\NormalTok{ (}\DecValTok{1}\SpecialCharTok{|}\NormalTok{subject), }\AttributeTok{data =}\NormalTok{lau\_exp, }\AttributeTok{family =} \StringTok{\textquotesingle{}binomial\textquotesingle{}}\NormalTok{)}
\end{Highlighting}
\end{Shaded}

\begin{enumerate}
\def\labelenumi{\roman{enumi}.}
\setcounter{enumi}{3}
\tightlist
\item
  finally, fit a model that models the interaction between \emph{task}
  and \emph{pas} and their main effects
\end{enumerate}

\begin{Shaded}
\begin{Highlighting}[]
\NormalTok{last\_one4 }\OtherTok{\textless{}{-}} \FunctionTok{glmer}\NormalTok{(correct}\SpecialCharTok{\textasciitilde{}}\NormalTok{task}\SpecialCharTok{*}\NormalTok{pas }\SpecialCharTok{+}\NormalTok{ (}\DecValTok{1}\SpecialCharTok{|}\NormalTok{subject), }\AttributeTok{data =}\NormalTok{lau\_exp, }\AttributeTok{family =} \StringTok{\textquotesingle{}binomial\textquotesingle{}}\NormalTok{)}
\end{Highlighting}
\end{Shaded}

\begin{enumerate}
\def\labelenumi{\alph{enumi}.}
\setcounter{enumi}{21}
\tightlist
\item
  describe in your words which model is the best in explaining the
  variance in accuracy
\end{enumerate}

\begin{Shaded}
\begin{Highlighting}[]
\NormalTok{accuracyfun }\OtherTok{\textless{}{-}} \ControlFlowTok{function}\NormalTok{(model, data) \{}
\NormalTok{  predicted }\OtherTok{\textless{}{-}} \FunctionTok{predict}\NormalTok{(model, data, }\AttributeTok{type=}\StringTok{\textquotesingle{}response\textquotesingle{}}\NormalTok{)}
\NormalTok{  predicted }\OtherTok{\textless{}{-}} \FunctionTok{ifelse}\NormalTok{(predicted }\SpecialCharTok{\textgreater{}} \FloatTok{0.5}\NormalTok{, }\DecValTok{1}\NormalTok{, }\DecValTok{0}\NormalTok{)}
\NormalTok{  tab }\OtherTok{\textless{}{-}}  \FunctionTok{table}\NormalTok{(data}\SpecialCharTok{$}\NormalTok{correct, predicted)}
\NormalTok{  accuracy }\OtherTok{\textless{}{-}}\NormalTok{ (tab[}\DecValTok{1}\NormalTok{,}\DecValTok{1}\NormalTok{] }\SpecialCharTok{+}\NormalTok{ tab[}\DecValTok{2}\NormalTok{,}\DecValTok{2}\NormalTok{])}\SpecialCharTok{/}\NormalTok{(tab[}\DecValTok{1}\NormalTok{,}\DecValTok{1}\NormalTok{] }\SpecialCharTok{+}\NormalTok{ tab[}\DecValTok{2}\NormalTok{,}\DecValTok{2}\NormalTok{] }\SpecialCharTok{+}\NormalTok{tab[}\DecValTok{1}\NormalTok{,}\DecValTok{2}\NormalTok{] }\SpecialCharTok{+}\NormalTok{ tab[}\DecValTok{2}\NormalTok{,}\DecValTok{1}\NormalTok{])}
  \FunctionTok{return}\NormalTok{(accuracy)}
  
\NormalTok{\}}

\FunctionTok{accuracyfun}\NormalTok{(last\_one2, lau\_exp)}
\end{Highlighting}
\end{Shaded}

\begin{verbatim}
## [1] 0.7421775
\end{verbatim}

\begin{Shaded}
\begin{Highlighting}[]
\FunctionTok{accuracyfun}\NormalTok{(last\_one3, lau\_exp)}
\end{Highlighting}
\end{Shaded}

\begin{verbatim}
## [1] 0.7421775
\end{verbatim}

\begin{Shaded}
\begin{Highlighting}[]
\FunctionTok{accuracyfun}\NormalTok{(last\_one4, lau\_exp)}
\end{Highlighting}
\end{Shaded}

\begin{verbatim}
## [1] 0.7409004
\end{verbatim}

\end{document}
